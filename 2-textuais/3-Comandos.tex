\chapter{COMANDOS LATEX}
\label{cap:comandos}

Este capítulo tem por objetivo descrever os comandos \textit{LaTex} para a confecção da monografia. A \autoref{sec:divisao} descreve como é a divisão de documentos em conformidade com a ABNT NBR 6024 \cite{NBR6024:2012}. 

Consulte os manuais do \abnTeX\
\cite{abntex2classe,abntex2cite,abntex2cite-alf} para maiores detalhes sobre o uso deste template. Exemplos e informações adicionais de uso de \abnTeX\ e de \LaTeX podem ser encontrados em \citeonline{abntex2modelo}.

\section{DIVISÃO DE DOCUMENTOS}
\label{sec:divisao}

A principal divisão de uma monografia é chamada de seção primária (conhecida por capítulo)\cite{NBR6024:2012}. Para criar um capítulo, deve-se utilizar o comando \verb!\chapter{}!. Por exemplo, este capítulo foi definido conforme o comando:

\begin{verbatim}
    \chapter{COMANDOS LATEX}
\end{verbatim}

Um capítulo é dividido em seções, chamadas de secundárias. Uma seção secundária é criada utilizando o comando

\begin{verbatim}
    \section{<nome da seção>} 
\end{verbatim}

Da mesma maneira que um capítulo é referenciado, qualquer seção (não importa o seu nível) pode ser referenciada utilizando o \verb!autoref{}!

% ---
\subsection{Seção Terciária}
\label{sec:secaoTerciaria}
% ---

 Uma seção secundária pode ser dividida em mais seções, chamadas de terciária. Esta é a \autoref{sec:secaoTerciaria}. Uma seção terciária é criada utilizando o comando 

\begin{verbatim}
    \subsection{<nome da subseção>} 
\end{verbatim}

\subsubsection{Seção Quaternária}
\label{sec:secaoQuaternaria}

A subdivisão do texto a partir de uma seção terciária é chamada de seção quaternária. Uma seção quaternária é criada utilizando o comando 

\begin{verbatim}
    \subsubsection{<nome da subseção>} 
\end{verbatim}

Não se recomenda a divisão do texto a partir de uma seção quaternária. 

\section{REFERENCIAR SEÇÕES, CAPÍTULOS E OUTROS ELEMENTOS}

Em um texto técnico, é muito comum que capítulos ou seções sejam referenciados. Além de existir uma formatação própria para realizar o referenciamento, deve-se garantir que a numeração esteja sempre correta (a numeração das divisões do documento podem mudar durante a sua construção). 

De modo a eliminar qualquer dúvida sobre os termos a serem utilizados ao referenciar, somente as palavras apêndice, anexo, capítulo e seção devem ser utilizadas para referenciar qualquer divisão do documento. Não existe lugar para o uso de subseção ou troca do termo capítulo por seção, ou vice-versa. Além disso, como o termo é seguido por um número identificador, a primeira letra deve estar em maiúscula. 

Para que a divisão do documento seja referenciada conforme as normas, deve-se adicionar o comando de definição de rótulo \verb!\label{}!. Por exemplo, este capítulo foi definido conforme os comandos:

\begin{verbatim}
     \chapter{COMANDOS LATEX}
     \label{cap:comandos}
\end{verbatim}

Para referenciar uma divisão, utiliza-se o comando \verb!autoref{}! com o rótulo designado. Além de realizar uma referência ao número da divisão, o comando produz o termo apropriado (depende do tipo de título). 

No caso deste capítulo, o comando \verb!\autoref{cap:comandos}! produzirá o texto \autoref{cap:comandos}. Nota-se que o capítulo deve possuir um rótulo definido como "cap:estrutura". No caso de seções, o texto produzido é "\autoref{sec:secaoTerciaria}." ou "\autoref{sec:secaoQuaternaria}." Em todos os casos, o termo é gerado com a primeira letra maiúscula conforme a formatação da  \ac{ABNT}. 

No caso de múltiplas seções serem referenciadas, deve-se utilizar o termo da divisão seguido pelo comando \verb!\ref{}! para a inserção do número. Por exemplo,

"As Seções \ref{sec:enum} e \ref{sec:divisao} tratam..." 

"Os Capítulos \ref{cap:configuracao} e \ref{cap:consideracoes} apresentam..."

É importante que o termo tenha a primeira letra em maiúsculo. 


\section{ILUSTRAÇÕES}

As ilustrações incluem figuras, tabelas, quadros e algoritmos. Quando utilizadas, estas ilustrações produzem uma lista na parte pré-textual do documento. 

\subsection{Figuras}

As ilustrações devem sempre ter numeração contínua e única em todo o documento:

\begin{citacao}
	Qualquer que seja o tipo de ilustração, sua identificação aparece na parte
	superior, precedida da palavra designativa (desenho, esquema, fluxograma,
	fotografia, gráfico, mapa, organograma, planta, quadro, retrato, figura,
	imagem, entre outros), seguida de seu número de ordem de ocorrência no texto,
	em algarismos arábicos, travessão e do respectivo título. Após a ilustração, na
	parte inferior, indicar a fonte consultada (elemento obrigatório, mesmo que
	seja produção do próprio autor), legenda, notas e outras informações
	necessárias à sua compreensão (se houver). A ilustração deve ser citada no
	texto e inserida o mais próximo possível do trecho a que se
	refere \cite{NBR14724:2011}.
\end{citacao}

A \autoref{fig:exemploFigura} mostra um exemplo de figura, com os principais elementos descritos pela norma. A referência a uma figura deve utilizar o comando \verb!autoref{}!. O comando \texttt{UCSfigura} garante o alinhamento da fonte no lado esquerdo da figura.  

    \begin{figure}[!ht]
    	\caption{Exemplo de figura}
    	\centering 
    	\label{fig:exemploFigura}
    	\UCSfigura {}{ 
    		\includegraphics{figuras/estruturaTCC.png}
    		}{
    		\fonte{Adaptado de \citeonline{guiaUCS}.}
    	}
    \end{figure}   

Algumas das figuras podem ser formadas por várias figuras e as mesmas referenciadas no texto. O pacote \texttt{subfigure}  permite inserir e identificar cada uma destas figuras. A \autoref{fig:estrutura} mostra um exemplo de como colocar duas figuras lado a lado com rótulos próprios para referenciamento no texto. A \autoref{fig:estuturaABNT1} aponta para a figura da esquerda e \autoref{fig:estuturaTCC1} a da direita. 


\begin{figure}[!ht]
	\centering
	\caption{Estruturas de trabalhos pela ABNT e o adotado pela Área da Informática da UCS.}
	\label{fig:estrutura} 
	\begin{subfigure}{0.45\textwidth}
		\centering
		\includegraphics[width = \textwidth]{figuras/estruturaABNT.png}
		\caption{Estrutura de Documentos conforme a ABNT}
		\label{fig:estuturaABNT1}
	\end{subfigure}
	\begin{subfigure}{0.45\textwidth}
		\centering
		\includegraphics[width = \textwidth]{figuras/estruturaTCC.png}
		\caption{Estrutura do TCC }
		\label{fig:estuturaTCC1}
	\end{subfigure}
			
	\fonte{O Autor (2021).}
		
\end{figure}

O \autorefanexo{anex:exemploFiguras} apresenta mais exemplos de figuras que utilizam o \texttt{subfigure}. 

No caso de imagens retiradas da internet, não é necessário adicionar a URL do site na lista de referências do trabalho. Como o objetivo é somente ilustrar, pode-se referenciar o website. A \autoref{fig:exemploImagemInternet} mostra um exemplo de citação da fonte de internet. 

\begin{figure}[!ht]
	\caption{Exemplo de figura com fonte da internet.}
	\centering 
	\label{fig:exemploImagemInternet}
	\UCSfigura {}{ 
		\includegraphics[width = 0.45\textwidth]{figuras/biblio-central.jpg}
		}{ 	\fonte{Disponível em: <https://www.ucs.br/site/biblioteca/biblioteca-central/>. Acesso em 21 mai. 2021} 
		}
\end{figure} 

% ---
\subsection{Tabelas}
% ---
As tabelas apresentam informações numéricas tratadas
estatisticamente \cite{ibge1993}. A estrutura da tabela é constituída de traços (retas perpendiculares) e delimitada, na parte superior e na parte inferior por traços horizontais paralelos. Não delimitar (fechar) por traços verticais os extremos da tabela à direita e à esquerda. A \autoref{tab:ibge} é um exemplo de tabela conforme o padrão IBGE requerido pelas normas da ABNT para documentos técnicos e acadêmicos. Além da fonte na parte inferior da tabela, o exemplo traz como se insere uma nota e anotações, quando necessários. O comando \texttt{IBGEtab} garante o alinhamento da fonte no lado esquerdo da tabela.  

\begin{table}[!ht]
	\caption{\label{tab:ibge} Um Exemplo de tabela alinhada que pode ser longa ou curta}%
	\IBGEtab{}{
		\begin{tabular}{lccc}
			\toprule
			\textbf{Pessoa} & \textbf{Idade} & \textbf{Peso} & \textbf{Altura} \\
			\midrule \midrule
			Marcos          & 26             & 68            & 178             \\
			Ivone           & 22             & 57            & 162             \\ 
			...             & ...            & ...           & ...             \\
			Sueli           & 40             & 65            & 153             \\ 
			\bottomrule
		\end{tabular}
		}{
		\fonte{O Autor (2021).}
		\nota{esta é uma nota, que diz que os dados são baseados na
		regressão linear.}
		\nota[Anotações]{uma anotação adicional, seguida de várias outras.}
	}
\end{table}
    
O \autoref{cap:exemploTabelas} apresenta mais exemplos de tabelas, incluindo uma tabela que ultrapassa uma página (formatação própria). 

% ---
\subsection{Quadros}
% ---

A formatação do quadro é similar à tabela, mas deve ter suas laterais fechadas e conter as linhas horizontais. O \autoref{quad:exemplo} mostra um exemplo do uso de quadro no texto. O comando \texttt{UCSQuadro} garante o alinhamento da fonte no lado esquerdo do quadro.   

\begin{quadro}[!ht]
	\caption{\label{quad:exemplo} Exemplo de quadro}
	\UCSQuadro{}{%
	\begin{tabular}{|c|l|c|}
		\hline
		\textbf{Norma} & \textbf{Título} & \textbf{Ano de Publicação} \\
		\hline
		NBR 14724  & Trabalho Acadêmico    & 2011  \\ \hline
		NBR 10520   & Citações   & 2002   \\ \hline
		NBR 6023  & Referências & 2018 (Versão Corrigida 2:2020)  \\ \hline
		NBR 6027   & Sumário    & 2012  \\ \hline
		NBR 6028  & Resumo e Abstract     & 2003   \\ \hline
		NBR 6024  & Numeração progressiva das seções    & 2012 \\ \hline
		NBR 6034  & Índice    & 2004   \\ \hline
	\end{tabular}
	}{
	\fonte{O Autor (2021).
	}
}\end{quadro}

O \autoref{cap:exemploQuadros} apresenta mais exemplos de quadros, incluindo um quadro que ultrapassa uma página (formatação própria). 

\subsection{Algoritmos/Códigos Fontes}    
    
Não existe uma norma que padronize a formatação de algoritmos ou código fontes em trabalhos técnicos. Neste modelo, utiliza-se o pacote \textit{Listings} \footnote{https://ctan.org/pkg/listings} para a formatação. 

Para facilitar a utilização deste recurso, a linguagem padrão de formatação é C, conforme consta no arquivo TCC.tex e ilustrado na  \ref{fig:linguagem}. Caso não seja C a linguagem mais utilizada, substitua pelo nome da linguagem a ser utilizada conforme o padrão definido na documentação do pacote \textit{Listings}.

\begin{figure}[!ht]
\centering 
\caption{Configuração do pacote \textit{Listings}}
\label{fig:linguagem}
\UCSfigura{}{%
    \includegraphics[scale=0.70]{figuras/linguagem.png}
}{%
 \fonte{O Autor (2021).}%
 }
\end{figure}

Um exemplo é apresentado no Algoritmo \ref{codigo:exemplo1}. Nota que para a configuração deste algoritmo, foram utilizadas somente 2 campos: \textit{caption} e \textit{label}. 

\begin{lstlisting}[ caption={Exemplo de código em C},
                    label={codigo:exemplo1}]
#include <stdio.h>
int main() {    
    int numero1, numero2, soma;
    
    printf("Digite 2 numeros: ");
    scanf("%d %d", &numero1, &numero2);

    // Calcular a soma
    soma = numero1 + numero2;      
    
    printf("%d + %d = %d", numero1, numero2, soma);
    return 0;
}
\end{lstlisting}
\fonte{O Autor (2021)}

No caso de precisar adicionar um código que não seja da linguagem padrão, pode-se fazer utilizando o campo \textit{language} ( \verb!\begin{lstlisting}[language=Python,...]!). Um exemplo de um código na linguagem Python é mostrado no Algoritmo \ref{codigo:exemplo2}).

\begin{lstlisting}[language=Python,
                    caption={Exemplo de código em Python},
                    label={codigo:exemplo2}]
n = int(input("Digite o valor de n: "))
fatorial = 1
while (n > 0):
    fatorial = fatorial * n
    n -= 1
print(fatorial)
\end{lstlisting}
\fonte{O Autor (2021)}

\section{SIGLAS}
\label{sec:siglas}

Para utilizar siglas, primeiro defina a sigla no arquivo "acronimos.tex"~ na pasta "1-pre-textuais"~com o comando 

 \begingroup
\fontsize{10pt}{12pt}\selectfont
   \verb!\acro{PC}{Personal Computer}!
\endgroup
    
O comando deve ser colocado no arquivo de tal maneira que mantenha a ordem alfabética. O \textit{LaTex} garante somente se ela deve ser mostrada ou não na lista.

Para referenciar a sigla, deve-se utilizar o comando:

\begingroup
\fontsize{10pt}{12pt}\selectfont
\verb!\ac{PC}! \newline
\endgroup

A primeira vez que o comando é usado para uma determinada sigla, aparece o significado por extenso da sigla com a sua abreviação em seguida, desta maneira --- \ac{PC}. A partir da segunda vez que o comando para uma determinada sigla é usado, aparace apenas a sigla --- \ac{PC}. Assim, o autor nunca terá que se preocupar com a regra de sempre colocar a sigla por extenso na primeira vez. 

No caso de utilizar o plural da sigla, utilize-o com a letra s após a referência, como em \acp{PC}. Para referenciar a sigla no plural, deve-se utilizar o comando:

\begingroup
\fontsize{10pt}{12pt}\selectfont
\verb!\acp{PC}! \newline
\endgroup

É muito comum ver em trabalhos o uso de 's após a sigla. Este tipo de construção é utilizada para o possessivo da língua inglesa. 

\textbf{Não use siglas nos títulos e evite fortemente usar no Resumo. }

% ---
\section{ENUMERAÇÕES: ALÍNEAS e SUB-ALÍNEAS}
\label{sec:enum} 
% ---

Quando for necessário enumerar os diversos assuntos de uma seção que não possua título, esta deve ser subdividida em alíneas \cite[4.2]{NBR6024:2012}:

\begin{alineas}
  \item os diversos assuntos que não possuam título próprio, dentro de uma mesma
  seção, devem ser subdivididos em alíneas; 
  
  \item o texto que antecede as alíneas termina em dois pontos;
  \item as alíneas devem ser indicadas alfabeticamente, em letra minúscula,
  seguida de parêntese. Utilizam-se letras dobradas, quando esgotadas as
  letras do alfabeto;

  \item as letras indicativas das alíneas devem apresentar recuo em relação à
  margem esquerda;

  \item o texto da alínea deve começar por letra minúscula e terminar em
  ponto-e-vírgula, exceto a última alínea que termina em ponto final;

  \item o texto da alínea deve terminar em dois pontos, se houver subalínea;

  \item a segunda e as seguintes linhas do texto da alínea começa sob a
  primeira letra do texto da própria alínea;
  
  \item subalíneas \cite[4.3]{NBR6024:2012} devem ser conforme as alíneas a
  seguir:

  \begin{alineas}
     \item as subalíneas devem começar por travessão seguido de espaço;

     \item as subalíneas devem apresentar recuo em relação à alínea;

     \item o texto da subalínea deve começar por letra minúscula e terminar em
     ponto-e-vírgula. A última subalínea deve terminar em ponto final, se não
     houver alínea subsequente;

     \item a segunda e as seguintes linhas do texto da subalínea começam sob a
     primeira letra do texto da própria subalínea.
  \end{alineas}
  
  \item no \abnTeX\ estão disponíveis os ambientes \texttt{incisos} e
  \texttt{subalineas}, que em suma são o mesmo que se criar outro nível de
  \texttt{alineas}, como nos exemplos a seguir:
  
  \begin{incisos}
    \item \textit{Um novo inciso em itálico};
  \end{incisos}
  
  \item Alínea em \textbf{negrito}:
  
  \begin{subalineas}
    \item \textit{Uma subalínea em itálico};
    \item \underline{\textit{Uma subalínea em itálico e sublinhado}}; 
  \end{subalineas}
  
  \item Última alínea com \emph{ênfase}.
  
\end{alineas}

É possível modificar a enumeração para utilizar outras sequências. Deve-se colocar o novo símbolo entre colchetes após o comando \verb=\item[]=. Um exemplo de enumeração alterando a sequência pode ser definido da seguinte forma:

\begin{alineas}
	\item[1.] primeiro item
	\item[2.] segundo item
	\begin{alineas}
		\item[a)] primeiro subalínea 
		\item[b)] segundo subalínea
	\end{alineas}
	\item[3.] terceiro item
\end{alineas}

Existem diversos símbolos pré-definidos que podem ser utilizados para a enumeração:

\begin{alineas}
	\item[$\times$] primeiro item
	\item[$\bullet$] segundo item
	\item[$\diamond$] terceiro item
	\item[$\cdot$] quarto item
	\item[$\ast$] quinto item
	\item[$\circ$] sexto item
	\item[$-$] sétimo item
\end{alineas}

\section{CITAÇÕES} 

As citações são menções de informações extraídas de outras fontes \cite{NBR10520:2002}. Elas podem ser uma transcrição textual de parte da obra (citação direta) ou um texto que transcreve a informação com as palavras do autor do trabalho (citação indireta). 
No caso de citações indiretas, existem duas formas de realizar a citação:

\begin{itemize}
	\item \textbf{Início ou meio da frase}: deve-se citar o autor com a primeira letra em maiúsculo e colocar em seguida, entre parênteses o ano. Para esta construção, utiliza-se o comando \verb=\citeonline{}=. Por exemplo,  \citeonline{Acevedo2013}. No caso de querer adicionar o intervalo de páginas, o comando é  \verb=\citeonline[p. 10--15]{key}=. Por exemplo, \citeonline[pp. 21--30]{Acevedo2013}. No caso de somente uma página, o comando é  \verb=\citeonline[p. 10]{key}=. Por exemplo, \citeonline[pp. 21]{Acevedo2013}.
	          
	\item \textbf{Final da frase}: deve-se citar entre parênteses o autor em letras maiúsculas e colocar em seguida, entre parênteses o ano. Para esta construção, utiliza-se o comando \verb=\cite{}=. Por exemplo, \cite{Acevedo2013}. No caso de querer adicionar o intervalo de páginas, o comando é \verb=\cite[p. 10--15]{key}=. Por exemplo, \cite[pp. 21--30]{Acevedo2013}. No caso de somente uma página, o comando é  \verb=\cite[p. 10]{key}=. Por exemplo, \cite[p. 21]{Acevedo2013}.
\end{itemize}

Para realizar a citação direta de um texto longo (mais de 3 linhas), deve-se utilizar o seguinte comando: \verb=\begin{citacao} trecho transcrito \end{citacao}=. Por exemplo, 
\begin{citacao}
	As citações diretas, no texto, com mais de três linhas, devem ser destacadas com recuo de 4 cm da margem esquerda, com letra menor que a do texto utilizado e sem as aspas. No caso de documentos datilografados, deve-se observar apenas
	o recuo \cite[p. 2]{NBR10520:2002}.
\end{citacao}

No caso de não ter acesso à publicação original, é possível realizar a citação de uma citação, ou seja, a transcrição direta ou indireta de um texto ao qual não se teve acesso ao original. Existe uma formatação própria para realizar este tipo de citação utilizando o termo apud ("citado por"). A citação é composta pelo nome do autor original (ao qual não se teve acesso à obra e não deve constar nas referências do trabalho), o termo apud e o nome do autor da obra consultada (a qual deve constar nas referências do trabalho). Por exemplo, no Guia de Trabalhos Acadêmicos da UCS (\cite{guiaUCS}) existe a menção da norma 12225 da ABNT. A sua citação pode ser realizada da seguinte maneira:

\begingroup
\begin{adjustwidth}{2cm}{2cm}
    \noindent
	As informações devem ser impressas conforme a norma ABNT NBR 12225:2004 (\citeauthor{nbr12225:2004}, \citeyear{nbr12225:2004}\footnote{ASSOCIAÇÃO BRASILEIRA DE NORMAS TÉCNICAS.\textbf{NBR 10520}: Informação e documentação — lombada — apresentação. Rio de Janeiro, 2004. 3 p.} apud \citeauthor{guiaUCS}, \citeyear{guiaUCS}). 
	
	\centering ou
	
	\justifying
	\noindent\citeauthor{nbr12225:2004}\footnote{ASSOCIAÇÃO BRASILEIRA DE NORMAS TÉCNICAS.\textbf{NBR 10520}: Informação e documentação — lombada — apresentação. Rio de Janeiro, 2004. 3 p.} ( \citeyear{nbr12225:2004} apud \citeauthor{guiaUCS}, \citeyear{guiaUCS}) define as normas de como as informações devem ser impressas na lombada. 
\end{adjustwidth}
\endgroup

Em ambos os exemplos, o texto da referência original deve ser colocada no rodapé. De modo a reduzir a digitação, os dados da referência original podem ser colocados no arquivo de bibliografias (referencias.bib), mas utilizando o identificador @hidden (para que não faça parte da lista de referências):

\begingroup
\fontsize{10pt}{12pt}\selectfont
\begin{verbatim}
      @hidden{NBR12225:2004, 
          Address = {Rio de Janeiro},  
          Organization = {ASSOCIAÇÃO BRASILEIRA DE NORMAS TÉCNICAS}, 
          Pages = 3,  
          Subtitle = {Informação e documentação - lombada - apresentação},
          Title = {{NBR} 12225}, 
          Year = 2004
       }  
\end{verbatim}
\endgroup

% ---
\section{EXPRESSÕES MATEMÁTICAS}
\label{sec:expressoes}
% ---

As expressões matemáticas devem ser escritas utilizando ambiente \texttt{equation} para serem numeradas (para que sejam referenciadas):

\begin{equation}\label{eq:exemplo1}
\hat{c} = \underset{c}{\arg\max}  P(\textbf{x}|c).
\end{equation}

A referência a uma equação deve utilizar o comando \verb!autoref{}!. A \autoref{eq:exemplo1} é um exemplo de referência. 

No caso de uma expressão não necessitar ser numerada, é possível usar colchetes para indicar o início de uma expressão:

\[
\left|\sum_{i=1}^n a_ib_i\right|
\le
\left(\sum_{i=1}^n a_i^2\right)^{1/2}
\left(\sum_{i=1}^n b_i^2\right)^{1/2}
\]

Se a expressão matemática puder estar na mesma linha do texto, pode-se escrever expressões matemáticas entre cifrões (\$), como em $ \lim_{x \to \infty} \exp(-x) = 0 $.

Consulte mais informações sobre expressões matemáticas em
\url{https://github.com/abntex/abntex2/wiki/Referencias}.

\section{FORMATAÇÃO DO FONTE}

Para uma formatação do \textit{LaTex}, pode-se utilizar o formatador online do C. Albert Thompson  \footnote{https://c.albert-thompson.com/latex-pretty/}. Este site realiza a indentação do código-fonte \textit{LaTex}. 

