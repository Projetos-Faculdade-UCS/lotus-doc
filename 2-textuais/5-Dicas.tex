\chapter{DICAS GERAIS}
\label{cap:dicas}
	
Este capítulo tem por objetivo fornecer orientações para a construção da monografia. A \autoref{sec:capitulos} descreve algumas orientações sobre a construção dos capítulos. 

\section{CAPÍTULOS} 
\label{sec:capitulos} 

Todo trabalho de conclusão possui os elementos textuais Introdução e Conclusões. A questão que resta é como estruturar o trabalho entre estes dois capítulos. Nesta seção, serão discutidos algumas orientações sobre como um capítulo pode ser construído. 

Em uma monografia, espera-se que o texto esteja organizado de tal forma que o leitor consiga acompanhar e compreender o que está sendo feito e porquê. Assim, geralmente o desenvolvimento está distribuído em capítulos que tratem do referencial teórico, proposta e experimentos/resultados/discussão. Este último aparece na monografia da segunda disciplina de trabalho de conclusão. 

Em algumas situações, o referencial teórico pode aparecer em dois capítulos devido a sua complexidade ou amplitude. Por exemplo, pode ser que um capítulo forneça uma visão geral da área e as respectivas definições enquanto o o outro trate das características do problema específico do trabalho. O objetivo de dividir em mais de um capítulo é permitir um texto mais conciso e não maçante para o leitor. O texto deve sempre considerar que a compreensão do texto pode ser gradual e acumulativa. 

Um capítulo é formado por texto que apresenta algum tipo de relação. Por exemplo, em um capítulo de referencial teórico, espera-se que o seu conteúdo seja formado por definições e relações com outros trabalhos afins. Em um capítulo que apresente uma proposta de solução, o texto deve tratar somente dos elementos desta solução. Para facilitar a leitura, inicie o capítulo explicando o motivo ou objetivo do mesmo. Além disso, pode-se descrever a estrutura dos capítulos apresentando cada uma das seções secundárias (como é feito no início deste capítulo). Assim, evita-se o uso de frases que apresentam a próxima seção. 

Uma seção que é geralmente incluída no final dos capítulos é a de considerações finais. Esta seção tem por objetivo desenhar conclusões sobre o que foi tratado no texto. Ela auxilia na leitura, pois traz os principais pontos do capítulo que serão considerados no(s) próximo(s) capítulo(s). Mas um erro que muitos cometem nesta seção é a de introduzir o que será tratado no próximo capítulo. É necessário lembrar que a seção de estrutura do trabalho na introdução e o texto inicial do capítulo já realizam este papel de explicar o que cada capítulo traz. 

\section{PARÁGRAFOS}

O parágrafo é a unidade de discurso de um texto \cite{martins2019}. É a divisão do texto em partes menores, que determinam um enfoque. Portanto, uma seção deve ser composta por parágrafos e não frases soltas. 

O parágrafo possui partes distintas\cite{martins2019}. O tópico frasal é a ideia principal do parágrafo. Exprime a ideia principal do parágrafo. Em seguida, frases que dão suporte ou explicam em maiores detalhes o tópico frasal são apresentadas. É possível ainda ter uma conclusão que resuma o parágrafo ou exponha o ponto de maior interesse ou um elemento relacionador que estabeleça o encadeamento entre os parágrafos. Toda vez que trocar o enfoque do parágrafo, deve-se criar um parágrafo. 

Como a primeira frase de um parágrafo estabelece a principal ideia de cada parágrafo, deve ser possível compreender o texto da seção a partir de cada frase. As demais frases de cada parágrafo devem somente dar mais detalhes sobre a ideia principal. Por isso, um parágrafo não deve ser iniciado apresentando uma ilustração.  

\section{APÊNDICES X ANEXOS}

Uma confusão muito comum em trabalhos é o uso de apêndices e anexos. O apêndice "é o texto ou documento elaborado pelo próprio autor, de modo a complementar o texto principal e é apresentado no final do trabalho" \cite{guiaUCS}. O anexo é "destinado à inclusão de materiais não 
elaborados pelo próprio autor, como cópias de artigos, manuais, fôlderes, balancetes, etc., visando a dar suporte à argumentação, fundamentação, ilustração ou comprovação" \cite{guiaUCS}. 

O apêndice e o anexo são elementos opcionais. Somente os títulos (seções primárias) destes elementos devem constar no sumário. Caso tais elementos possuírem ilustrações, as mesmas não devem constar nas respectivas listas da monografia. 

\section{ILUSTRAÇÕES}

Uma boa forma de explicar algo é usando ilustrações (desenhos, esquemas, fluxogramas, fotografias, gráficos, mapas, organogramas, plantas, quadros, retratos e outros). Uma ilustração por si só não faz sentido em um texto, ou seja, a ilustração deve estar sempre relacionada a um tópico do texto. 

Toda ilustração deve ser citada no texto e explicada. O texto deve indicar para o leitor o momento certo para olhar ou analisar a ilustração. Por isso, em algumas situações, deve-se explicar a ilustração. 

Se você entender que as ilustrações estão para auxiliar na compreensão do texto, isso vai facilitar o que deve ser escrito no texto ao referenciar a ilustração. 

\section{EQUAÇÕES}

A primeira consideração na utilização de equações é que elas são parte do texto. Isto é, elas podem ser o meio ou o fim de uma frase e, portanto, deve ter a pontuação definida. Por exemplo, pode-se utilizar a fórmula da relatividade $E = mc^2$ como parte da frase (deve ser limitada por cifrões. Ou ainda, pode-se utilizá-las de forma separada (em nova linha) para dar destaque e também numerá-la: 
\begin{citacao}
Este método normaliza cada variável utilizando a equação
\begin{equation}\label{eq:exemplo2}
Z=\frac{X-\mu}{\sigma},
\end{equation}

onde $\mu$ e $\sigma$ são a média e o desvio padrão estimados a partir do conjunto de vetores de características denotado por $X$.

Com base no teorema descrito, a \autoref{eq:exemplo2} pode ser reescrita...

\end{citacao}

Nota-se que após a equação, utiliza-se a vírgula como continuação da frase. É importante que todas as variáveis da equação sejam explicadas antes ou depois dela. Para que o texto após a equação comece na margem esquerda, não deve haver uma linha em branco após a equação.

Deve-se manter uma utilização consistente de variáveis ao longo do texto. NUNCA reutilize variáveis para definições diferentes. 




