\chapter{CONSIDERAÇÕES FINAIS}


Este trabalho abordou o desenvolvimento de um software para centralização e gestão de dados acadêmicos, com o objetivo de melhorar a organização, acessibilidade e utilização do conhecimento gerado por pesquisas e projetos em instituições de ensino e pesquisa. Através da integração de metodologias ágeis como TDD e BDD, foi possível criar uma solução tecnológica robusta e adaptável, capaz de atender às necessidades específicas de diferentes usuários, desde estudantes até gestores acadêmicos.

A implementação de tecnologias modernas para o desenvolvimento do Front-end (React e Flutter) e Back-end (Python e Django), bem como o uso de bancos de dados em nuvem, proporcionou uma base sólida para a criação de um sistema escalável e seguro. A arquitetura do sistema, detalhada ao longo do trabalho, demonstrou como diferentes componentes podem interagir de maneira eficiente para garantir a integridade e a confiabilidade dos dados.

Os testes realizados ao longo do desenvolvimento garantiram a qualidade do software, validando suas funcionalidades e assegurando que o sistema atendesse aos requisitos especificados. A abordagem modular adotada permitiu uma maior flexibilidade na implementação e na manutenção do sistema, facilitando futuras expansões e melhorias.

Como principais contribuições, destacam-se:
\begin{itemize}
    \item A centralização de dados acadêmicos em um único repositório acessível, evitando a duplicação de esforços e promovendo a colaboração entre pesquisadores.
    \item A facilitação da combinação e análise de dados, potencializando a descoberta de novas soluções e insights.
    \item A promoção de um ambiente colaborativo e inovador, incentivando a integração entre diferentes plataformas e dispositivos.
\end{itemize}

Em conclusão, o desenvolvimento deste software representa um avanço significativo na gestão de dados acadêmicos, proporcionando uma plataforma eficiente e intuitiva que beneficia toda a comunidade acadêmica.



    

    