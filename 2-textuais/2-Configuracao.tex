\chapter{CONFIGURAÇÃO E ESTRUTURA DO TRABALHO}
\label{cap:configuracao}

Este capítulo tem por objetivo apresentar a estrutura do modelo do \ac{TCC} e descrever o processo de configuração do documento no editor online de documentos \textit{LaTex} Overleaf. A \autoref{sec:estruturaTC} apresenta a estrutura do trabalho de conclusão conforme a ABNT. A \autoref{sec:modeloTCC} descreve o modelo a ser utilizado e a sua estrutura. A \autoref{sec:primeiraConfiguracao} descreve os primeiros passos na configuração do trabalho. 

\section{ESTRUTURA DO TRABALHO DE CONCLUSÃO}
\label{sec:estruturaTC}
Conforme a norma ABNT NBR 14724:2011, a estrutura de teses, dissertações, \ac{TCC}, entre outros, compreende elementos obrigatórios e opcionais (e a sua respectiva ordem). Como o foco deste documento é \ac{TCC}, a Figura \ref{fig:estruturaTCC} apresenta os elementos obrigatórios e opcionais de um TCC. A ordem em que eles são apresentados deve ser mantido no documento. 

\begin{figure}[h]
	\caption{Estrutura do Trabalho de Conclusão para os cursos da Área da Informática}
	\centering 
	\label{fig:estruturaTCC}
	\UCSfigura {}{ 
	\includegraphics{figuras/estruturaTCC.png}
		}{
	\fonte{Adaptado de \citeonline{guiaUCS}.}
	}
\end{figure}    

Com o objetivo de simplificar o processo de confecção do \ac{TCC}, alguns elementos da estrutura não precisam constar na monografia submetida na primeira disciplina do \ac{TCC}. Tais elementos são mais apropriados para versão final da monografia na segunda e última disciplina de trabalho de conclusão de curso.  Os elementos são: 
\begin{itemize}
	\item Folha de aprovação
	\item Dedicatória
	\item Agradecimentos
	\item Epígrafe
	\item Resumo na língua estrangeira
\end{itemize}
Exceto a folha de aprovação, os demais elementos são opcionais e não precisam constar mesmo na monografia final.

É importante que o aluno utilize como base o guia de elaboração de trabalhos acadêmicos da UCS\cite{guiaUCS} para maiores detalhes.  

\section{MODELO DO TRABALHO DE CONCLUSÃO}
\label{sec:modeloTCC}

Para facilitar a formatação dos trabalhos de conclusão, este documento foi construído utilizando a ferramenta de edição de documentos \textit{LaTex} chamada \textit{Overleaf}\footnote{\url{http://overleaf.com}}. O \textit{Overleaf} é uma ferramenta \textit{online} que pode ser acessada por qualquer navegador de internet até mesmo de um \textit{smartphone}. Uma das vantagens desta ferramenta é a possibilidade de compartilhar o seu trabalho com o seu orientador. Assim, ele terá acesso imediato ao texto e realizar comentários acerca do trabalho. É necessário ter uma conta (grátis) em seu site para utilizá-lo. 

\subsection{Carregar o Modelo TCC} 
\label{subsec:loadTCC} 

Para começar o trabalho de edição, primeiramente é necessário o arquivo de projeto. O projeto nesta plataforma é um arquivo .zip. 

Após se conectar no site do \textit{Overleaf}, você deve realizar a carga do projeto. Clique em \textbf{Novo Projeto} e um menu de opções deve aparecer. Em seguida selecione a opção \textbf{Carregar Projeto}. Uma nova janela será aberta para você escolher ou arrastar um arquivo .zip. Utilize o projeto zipado disponibilizado pelo coordenador de \ac{TCC}. 

Altere o nome do projeto para facilitar a sua identificação. Uma sugestão é colocar TCC 1 ou 2 seguido pelo nome do aluno. Por exemplo, no caso do TCC 1 do aluno Carlos Heinz, o nome do projeto será TCC 1 - Carlos Heinz. 

\begin{tabular}{|l|c|} \hline
	DICA:  Carregue o projeto novamente para ter uma cópia de referência. 
                                                        \\\hline
\end{tabular}

\subsection{Estrutura do Projeto}
\label{sec:estruturaModelo} 

Ao abrir o projeto no Overleaf, a aba \textit{Project} (lateral esquerda) deve mostrar as pastas e arquivos, conforme ilustrado na \autoref{fig:modelo}:
\begin{alineas}
	\item 1-pre-textuais: pasta que contém os arquivos dos elementos pré-textuais: acrônimos, agradecimentos, folha de aprovação, dedicatória, epígrafe, resumos e lista de símbolos.
	\item 2-textuais: pasta que contém os arquivos dos elementos textuais (isto é, os capítulos da monografia --- da introdução à conclusão)
	\item 3-pos-textuais: pasta que contém os arquivos dos elementos pós-textuais: anexos, apêndices e a bibliografia.
	\item figuras: pasta que contém as figuras do trabalho.
	\item abntex2.cls: configuração da formatação (NÃO ALTERAR).
	\item TCC.tex: documento com as configurações do modelo. 
\end{alineas}

\begin{figure}[!ht]
	\caption{Aba \textit{Project} do modelo}
	\centering 
	\label{fig:modelo}
	\UCSfigura {}{ 
		\includegraphics{figuras/modelo.png}
		}{
		\fonte{O Autor (2021).}
	}
\end{figure}    

\subsection{Compartilhar o Projeto}
\label{sec:compartilhar} 

A versão grátis da ferramenta \textit{Overleaf} permite o compartilhamento do projeto com somente um usuário. No caso de um trabalho acadêmico, este compartilhamento poder ser feito com o orientador. 

Para compartilhar o projeto, deve-se estar editando o projeto. Clique no link \textbf{Compartilhar} no topo da página (lado direito). Uma janela será aberta para informar o e-mail de quem deve receber o compartilhamento. Em seguida, clique no botão compartilhar. 

\subsection{Atualizar o Projeto}
\label{sec:atualizar} 

O ambiente mostra na lateral direita uma versão compilada do documento para verificação. 

Para que cada alteração seja mostrada na versão compilada, deve-se clicar no botão Recompilar (acima do painel de visualização). 

No caso de alterar Para compartilhar o projeto, deve-se estar editando o projeto. Clique no link \textbf{Compartilhar} no topo da página (lado direito). Uma janela será aberta para informar o e-mail de quem deve receber o compartilhamento. Em seguida, clique no botão compartilhar. 


\section{PRIMEIRA CONFIGURAÇÃO}
\label{sec:primeiraConfiguracao} 

A primeira configuração tem por objetivo alimentar o projeto com as informações necessárias da monografia e atualizar a estrutura da monografia. 

\subsection{Identificações}

As informações de autoria, orientação e curso devem ser alteradas no arquivo TCC.tex, conforme ilustrado na \autoref{fig:info}. Informe os dados nos campos \texttt{titulo}, \texttt{autor}, \texttt{local}, \texttt{data} e \texttt{orientador}. No caso de não possuir um co-orientador, eliminar a linha do campo \texttt{coorientador} (linha 113).

A natureza do trabalho utilizada na folha de rosto e de aprovação é definida por meio do campo \texttt{preambulo}. Deve-se alterar somente o nome do curso (Ciência da Computação, Sistemas de Informação, Engenharia de Computação, ...). Para os alunos que são do campus da região dos Vinhedos, este campo pode incluir a identificação do campus: \textbf{Trabalho de Conclusão de Curso apresentado como requisito parcial à obtenção do título de Bacharel em Ciência da Computação na Área do Conhecimento de Ciências Exatas e Engenharias, Campus Universitário da Região dos Vinhedos, da Universidade de Caxias do Sul.  
}
\begin{figure}[!ht]
	\caption{Linhas do arquivo TCC.tex com as informações do trabalho}
	\centering 
	\label{fig:info}
	\UCSfigura {}{ 
		\includegraphics{figuras/info1.png}
		}{
		\fonte{O Autor (2021).}
	}
\end{figure}   

\subsection{Folha de Aprovação} 

Este elemento pré-textual deve ser inserido na monografia submetida na segunda disciplina de trabalho de conclusão. Duas informações devem ser preenchidas no arquivo aprovacao.tex na pasta 1-pre-textuais:

\begin{itemize}
    \item Data de Aprovação: deve ser preenchida com a data de apresentação de trabalho. Alterar o comando \verb!\large\textbf{Aprovado em 00/07/2021}!.
    \item Menbros da Banca: apesar do nome do orientador ser preenchido automaticamente, os membros da banca devem ser preenchidos manualmente. Os nomes dos membros da banca devem ser preenchidos com a respectiva titulação (Esp. --- Especialista, Me. ou Ma. --- Mestre, Dr. ou Dra. --- Doutor):
    \begingroup
    \fontsize{10pt}{12pt}\selectfont
    \begin{verbatim}
\assinatura{\ABNTEXchapterfont\large{Prof. Dra. Nome do Membro 1 da Banca} 
                                    \\ Universidade de Caxias do Sul - UCS}
\assinatura{\ABNTEXchapterfont\large{Prof. Dr. Nome do Membro 2 da Banca} 
                                    \\ Universidade de Caxias do Sul - UCS}
    \end{verbatim}
    \endgroup
\end{itemize}

\subsection{Elementos Opcionais} 

Alguns elementos do documento são opcionais ou não precisam necessariamente aparecer na monografia submetida na disciplina de Trabalho de Conclusão I \footnote{Verifique com a coordenação do seu curso sobre os itens obrigatórios que não precisam ser submetidos na respectiva disciplina}:


\begin{alineas}
	\item \textbf{Folha de aprovação}: este item é opcional na monografia submetida para a primeira disciplina de trabalho de conclusão, mas é obrigatória para aversão final na segunda e última disciplina. Para que este elemento não apareça, deve-se comentar as linhas no arquivo TCC.tex que o incluem:
	\begingroup
	\begin{verbatim}
    \newpage
    \include{1-pre-textuais/aprovacao}
	\end{verbatim}
	\endgroup
	\item \textbf{Dedicatória}: como este item é opcional, ele pode ser excluído do projeto. Nesse caso, deve-se apagar o arquivo dedicatoria.tex na pasta 1-pre-textuais e a linha no arquivo TCC.tex:
	\begingroup
	\begin{verbatim}
    \imprimirdedicatoria{1-pre-textuais/dedicatoria}
	\end{verbatim}
	\endgroup
	\item \textbf{Agradecimentos}: como este item é opcional, ele pode ser excluído do projeto. Nesse caso, deve-se apagar o arquivo agradecimentos.tex na pasta 1-pre-textuais e a linha no arquivo TCC.tex:
	\begingroup
	\begin{verbatim}
    \imprimiragradecimentos{1-pre-textuais/agradecimentos}
	\end{verbatim}
	\endgroup        
	\item \textbf{Epígrafe}: como este item é opcional, ele pode ser excluído do projeto. Nesse caso, deve-se apagar o arquivo epigrafe.tex na pasta 1-pre-textuais e a linha no arquivo TCC.tex:
	\begingroup
	\begin{verbatim}
    \include{1-pre-textuais/epigrafe}
	\end{verbatim}
	\endgroup   
	\item \textbf{Abstract}: o resumo na língua estrangeira é opcional. Por isso, ele pode ser excluído do projeto. Nesse caso, deve-se apagar o arquivo abstract.tex na pasta 1-pre-textuais e a linha no arquivo TCC.tex:.
	\begingroup
	\begin{verbatim}
    \include{1-pre-textuais/abstract}
	\end{verbatim}
	\endgroup   
	\item \textbf{Abreviaturas e Siglas}: como este item é opcional, ele pode ser excluído do projeto. No caso de exclusão, deve-se apagar o arquivo acronimos.tex na pasta 1-pre-textuais e a linha no arquivo TCC.tex:
	\begingroup
	\begin{verbatim}
    \include{1-pre-textuais/acronimos}
	\end{verbatim}
	\endgroup      
	\item \textbf{Símbolos}: como este item é opcional, ele pode ser excluído do projeto. No caso de exclusão, deve-se apagar o arquivo simbolos.tex na pasta 1-pre-textuais e a linha no arquivo TCC.tex:
	\begingroup
	\begin{verbatim}
    \include{1-pre-textuais/simbolos}
	\end{verbatim}
	\endgroup      
	\item \textbf{Apêndices}: se o trabalho não possui anexos, ele pode ser excluído do projeto. No caso de exclusão, deve-se apagar o arquivo anexos.tex na pasta 1-pre-textuais e a linha no arquivo TCC.tex:
	\begingroup
	\begin{verbatim}
    % ---
% Inicia os anexos
% ---
\begin{anexosenv}

	% ----------------------------------------------------------
	%\chapter{Anexo ...}
	%\label{anexo:nomea} 
	% ----------------------------------------------------------
	

	% ----------------------------------------------------------
	%\chapter{Anexo ...}
	%\label{anexo:nomeb} 
	% ----------------------------------------------------------

\end{anexosenv}

	\end{verbatim}
	\endgroup      
	\item \textbf{Anexos}: se o trabalho não possui anexos, ele pode ser excluído do projeto. No caso de exclusão, deve-se apagar o arquivo anexos.tex na pasta 1-pre-textuais e a linha no arquivo TCC.tex:
	\begingroup
	\begin{verbatim}
    % ---
% Inicia os anexos
% ---
\begin{anexosenv}

	% ----------------------------------------------------------
	%\chapter{Anexo ...}
	%\label{anexo:nomea} 
	% ----------------------------------------------------------
	

	% ----------------------------------------------------------
	%\chapter{Anexo ...}
	%\label{anexo:nomeb} 
	% ----------------------------------------------------------

\end{anexosenv}

	\end{verbatim}
	\endgroup
\end{alineas}

\subsection{Elementos Textuais}

Este projeto traz 6 capítulos, onde a introdução é o primeiro e o de conclusão o último. Caso o trabalho tenha menos capítulos, deve-se apagar a linha que o insere no documento no arquivo TCC.tex: 
 
\begingroup
\begin{verbatim}  
    \include{2-textuais/1-introducao}
    \chapter{CONFIGURAÇÃO E ESTRUTURA DO TRABALHO}
\label{cap:configuracao}

Este capítulo tem por objetivo apresentar a estrutura do modelo do \ac{TCC} e descrever o processo de configuração do documento no editor online de documentos \textit{LaTex} Overleaf. A \autoref{sec:estruturaTC} apresenta a estrutura do trabalho de conclusão conforme a ABNT. A \autoref{sec:modeloTCC} descreve o modelo a ser utilizado e a sua estrutura. A \autoref{sec:primeiraConfiguracao} descreve os primeiros passos na configuração do trabalho. 

\section{ESTRUTURA DO TRABALHO DE CONCLUSÃO}
\label{sec:estruturaTC}
Conforme a norma ABNT NBR 14724:2011, a estrutura de teses, dissertações, \ac{TCC}, entre outros, compreende elementos obrigatórios e opcionais (e a sua respectiva ordem). Como o foco deste documento é \ac{TCC}, a Figura \ref{fig:estruturaTCC} apresenta os elementos obrigatórios e opcionais de um TCC. A ordem em que eles são apresentados deve ser mantido no documento. 

\begin{figure}[h]
	\caption{Estrutura do Trabalho de Conclusão para os cursos da Área da Informática}
	\centering 
	\label{fig:estruturaTCC}
	\UCSfigura {}{ 
	\includegraphics{figuras/estruturaTCC.png}
		}{
	\fonte{Adaptado de \citeonline{guiaUCS}.}
	}
\end{figure}    

Com o objetivo de simplificar o processo de confecção do \ac{TCC}, alguns elementos da estrutura não precisam constar na monografia submetida na primeira disciplina do \ac{TCC}. Tais elementos são mais apropriados para versão final da monografia na segunda e última disciplina de trabalho de conclusão de curso.  Os elementos são: 
\begin{itemize}
	\item Folha de aprovação
	\item Dedicatória
	\item Agradecimentos
	\item Epígrafe
	\item Resumo na língua estrangeira
\end{itemize}
Exceto a folha de aprovação, os demais elementos são opcionais e não precisam constar mesmo na monografia final.

É importante que o aluno utilize como base o guia de elaboração de trabalhos acadêmicos da UCS\cite{guiaUCS} para maiores detalhes.  

\section{MODELO DO TRABALHO DE CONCLUSÃO}
\label{sec:modeloTCC}

Para facilitar a formatação dos trabalhos de conclusão, este documento foi construído utilizando a ferramenta de edição de documentos \textit{LaTex} chamada \textit{Overleaf}\footnote{\url{http://overleaf.com}}. O \textit{Overleaf} é uma ferramenta \textit{online} que pode ser acessada por qualquer navegador de internet até mesmo de um \textit{smartphone}. Uma das vantagens desta ferramenta é a possibilidade de compartilhar o seu trabalho com o seu orientador. Assim, ele terá acesso imediato ao texto e realizar comentários acerca do trabalho. É necessário ter uma conta (grátis) em seu site para utilizá-lo. 

\subsection{Carregar o Modelo TCC} 
\label{subsec:loadTCC} 

Para começar o trabalho de edição, primeiramente é necessário o arquivo de projeto. O projeto nesta plataforma é um arquivo .zip. 

Após se conectar no site do \textit{Overleaf}, você deve realizar a carga do projeto. Clique em \textbf{Novo Projeto} e um menu de opções deve aparecer. Em seguida selecione a opção \textbf{Carregar Projeto}. Uma nova janela será aberta para você escolher ou arrastar um arquivo .zip. Utilize o projeto zipado disponibilizado pelo coordenador de \ac{TCC}. 

Altere o nome do projeto para facilitar a sua identificação. Uma sugestão é colocar TCC 1 ou 2 seguido pelo nome do aluno. Por exemplo, no caso do TCC 1 do aluno Carlos Heinz, o nome do projeto será TCC 1 - Carlos Heinz. 

\begin{tabular}{|l|c|} \hline
	DICA:  Carregue o projeto novamente para ter uma cópia de referência. 
                                                        \\\hline
\end{tabular}

\subsection{Estrutura do Projeto}
\label{sec:estruturaModelo} 

Ao abrir o projeto no Overleaf, a aba \textit{Project} (lateral esquerda) deve mostrar as pastas e arquivos, conforme ilustrado na \autoref{fig:modelo}:
\begin{alineas}
	\item 1-pre-textuais: pasta que contém os arquivos dos elementos pré-textuais: acrônimos, agradecimentos, folha de aprovação, dedicatória, epígrafe, resumos e lista de símbolos.
	\item 2-textuais: pasta que contém os arquivos dos elementos textuais (isto é, os capítulos da monografia --- da introdução à conclusão)
	\item 3-pos-textuais: pasta que contém os arquivos dos elementos pós-textuais: anexos, apêndices e a bibliografia.
	\item figuras: pasta que contém as figuras do trabalho.
	\item abntex2.cls: configuração da formatação (NÃO ALTERAR).
	\item TCC.tex: documento com as configurações do modelo. 
\end{alineas}

\begin{figure}[!ht]
	\caption{Aba \textit{Project} do modelo}
	\centering 
	\label{fig:modelo}
	\UCSfigura {}{ 
		\includegraphics{figuras/modelo.png}
		}{
		\fonte{O Autor (2021).}
	}
\end{figure}    

\subsection{Compartilhar o Projeto}
\label{sec:compartilhar} 

A versão grátis da ferramenta \textit{Overleaf} permite o compartilhamento do projeto com somente um usuário. No caso de um trabalho acadêmico, este compartilhamento poder ser feito com o orientador. 

Para compartilhar o projeto, deve-se estar editando o projeto. Clique no link \textbf{Compartilhar} no topo da página (lado direito). Uma janela será aberta para informar o e-mail de quem deve receber o compartilhamento. Em seguida, clique no botão compartilhar. 

\subsection{Atualizar o Projeto}
\label{sec:atualizar} 

O ambiente mostra na lateral direita uma versão compilada do documento para verificação. 

Para que cada alteração seja mostrada na versão compilada, deve-se clicar no botão Recompilar (acima do painel de visualização). 

No caso de alterar Para compartilhar o projeto, deve-se estar editando o projeto. Clique no link \textbf{Compartilhar} no topo da página (lado direito). Uma janela será aberta para informar o e-mail de quem deve receber o compartilhamento. Em seguida, clique no botão compartilhar. 


\section{PRIMEIRA CONFIGURAÇÃO}
\label{sec:primeiraConfiguracao} 

A primeira configuração tem por objetivo alimentar o projeto com as informações necessárias da monografia e atualizar a estrutura da monografia. 

\subsection{Identificações}

As informações de autoria, orientação e curso devem ser alteradas no arquivo TCC.tex, conforme ilustrado na \autoref{fig:info}. Informe os dados nos campos \texttt{titulo}, \texttt{autor}, \texttt{local}, \texttt{data} e \texttt{orientador}. No caso de não possuir um co-orientador, eliminar a linha do campo \texttt{coorientador} (linha 113).

A natureza do trabalho utilizada na folha de rosto e de aprovação é definida por meio do campo \texttt{preambulo}. Deve-se alterar somente o nome do curso (Ciência da Computação, Sistemas de Informação, Engenharia de Computação, ...). Para os alunos que são do campus da região dos Vinhedos, este campo pode incluir a identificação do campus: \textbf{Trabalho de Conclusão de Curso apresentado como requisito parcial à obtenção do título de Bacharel em Ciência da Computação na Área do Conhecimento de Ciências Exatas e Engenharias, Campus Universitário da Região dos Vinhedos, da Universidade de Caxias do Sul.  
}
\begin{figure}[!ht]
	\caption{Linhas do arquivo TCC.tex com as informações do trabalho}
	\centering 
	\label{fig:info}
	\UCSfigura {}{ 
		\includegraphics{figuras/info1.png}
		}{
		\fonte{O Autor (2021).}
	}
\end{figure}   

\subsection{Folha de Aprovação} 

Este elemento pré-textual deve ser inserido na monografia submetida na segunda disciplina de trabalho de conclusão. Duas informações devem ser preenchidas no arquivo aprovacao.tex na pasta 1-pre-textuais:

\begin{itemize}
    \item Data de Aprovação: deve ser preenchida com a data de apresentação de trabalho. Alterar o comando \verb!\large\textbf{Aprovado em 00/07/2021}!.
    \item Menbros da Banca: apesar do nome do orientador ser preenchido automaticamente, os membros da banca devem ser preenchidos manualmente. Os nomes dos membros da banca devem ser preenchidos com a respectiva titulação (Esp. --- Especialista, Me. ou Ma. --- Mestre, Dr. ou Dra. --- Doutor):
    \begingroup
    \fontsize{10pt}{12pt}\selectfont
    \begin{verbatim}
\assinatura{\ABNTEXchapterfont\large{Prof. Dra. Nome do Membro 1 da Banca} 
                                    \\ Universidade de Caxias do Sul - UCS}
\assinatura{\ABNTEXchapterfont\large{Prof. Dr. Nome do Membro 2 da Banca} 
                                    \\ Universidade de Caxias do Sul - UCS}
    \end{verbatim}
    \endgroup
\end{itemize}

\subsection{Elementos Opcionais} 

Alguns elementos do documento são opcionais ou não precisam necessariamente aparecer na monografia submetida na disciplina de Trabalho de Conclusão I \footnote{Verifique com a coordenação do seu curso sobre os itens obrigatórios que não precisam ser submetidos na respectiva disciplina}:


\begin{alineas}
	\item \textbf{Folha de aprovação}: este item é opcional na monografia submetida para a primeira disciplina de trabalho de conclusão, mas é obrigatória para aversão final na segunda e última disciplina. Para que este elemento não apareça, deve-se comentar as linhas no arquivo TCC.tex que o incluem:
	\begingroup
	\begin{verbatim}
    \newpage
    \include{1-pre-textuais/aprovacao}
	\end{verbatim}
	\endgroup
	\item \textbf{Dedicatória}: como este item é opcional, ele pode ser excluído do projeto. Nesse caso, deve-se apagar o arquivo dedicatoria.tex na pasta 1-pre-textuais e a linha no arquivo TCC.tex:
	\begingroup
	\begin{verbatim}
    \imprimirdedicatoria{1-pre-textuais/dedicatoria}
	\end{verbatim}
	\endgroup
	\item \textbf{Agradecimentos}: como este item é opcional, ele pode ser excluído do projeto. Nesse caso, deve-se apagar o arquivo agradecimentos.tex na pasta 1-pre-textuais e a linha no arquivo TCC.tex:
	\begingroup
	\begin{verbatim}
    \imprimiragradecimentos{1-pre-textuais/agradecimentos}
	\end{verbatim}
	\endgroup        
	\item \textbf{Epígrafe}: como este item é opcional, ele pode ser excluído do projeto. Nesse caso, deve-se apagar o arquivo epigrafe.tex na pasta 1-pre-textuais e a linha no arquivo TCC.tex:
	\begingroup
	\begin{verbatim}
    \include{1-pre-textuais/epigrafe}
	\end{verbatim}
	\endgroup   
	\item \textbf{Abstract}: o resumo na língua estrangeira é opcional. Por isso, ele pode ser excluído do projeto. Nesse caso, deve-se apagar o arquivo abstract.tex na pasta 1-pre-textuais e a linha no arquivo TCC.tex:.
	\begingroup
	\begin{verbatim}
    \include{1-pre-textuais/abstract}
	\end{verbatim}
	\endgroup   
	\item \textbf{Abreviaturas e Siglas}: como este item é opcional, ele pode ser excluído do projeto. No caso de exclusão, deve-se apagar o arquivo acronimos.tex na pasta 1-pre-textuais e a linha no arquivo TCC.tex:
	\begingroup
	\begin{verbatim}
    \include{1-pre-textuais/acronimos}
	\end{verbatim}
	\endgroup      
	\item \textbf{Símbolos}: como este item é opcional, ele pode ser excluído do projeto. No caso de exclusão, deve-se apagar o arquivo simbolos.tex na pasta 1-pre-textuais e a linha no arquivo TCC.tex:
	\begingroup
	\begin{verbatim}
    \include{1-pre-textuais/simbolos}
	\end{verbatim}
	\endgroup      
	\item \textbf{Apêndices}: se o trabalho não possui anexos, ele pode ser excluído do projeto. No caso de exclusão, deve-se apagar o arquivo anexos.tex na pasta 1-pre-textuais e a linha no arquivo TCC.tex:
	\begingroup
	\begin{verbatim}
    % ---
% Inicia os anexos
% ---
\begin{anexosenv}

	% ----------------------------------------------------------
	%\chapter{Anexo ...}
	%\label{anexo:nomea} 
	% ----------------------------------------------------------
	

	% ----------------------------------------------------------
	%\chapter{Anexo ...}
	%\label{anexo:nomeb} 
	% ----------------------------------------------------------

\end{anexosenv}

	\end{verbatim}
	\endgroup      
	\item \textbf{Anexos}: se o trabalho não possui anexos, ele pode ser excluído do projeto. No caso de exclusão, deve-se apagar o arquivo anexos.tex na pasta 1-pre-textuais e a linha no arquivo TCC.tex:
	\begingroup
	\begin{verbatim}
    % ---
% Inicia os anexos
% ---
\begin{anexosenv}

	% ----------------------------------------------------------
	%\chapter{Anexo ...}
	%\label{anexo:nomea} 
	% ----------------------------------------------------------
	

	% ----------------------------------------------------------
	%\chapter{Anexo ...}
	%\label{anexo:nomeb} 
	% ----------------------------------------------------------

\end{anexosenv}

	\end{verbatim}
	\endgroup
\end{alineas}

\subsection{Elementos Textuais}

Este projeto traz 6 capítulos, onde a introdução é o primeiro e o de conclusão o último. Caso o trabalho tenha menos capítulos, deve-se apagar a linha que o insere no documento no arquivo TCC.tex: 
 
\begingroup
\begin{verbatim}  
    \include{2-textuais/1-introducao}
    \chapter{CONFIGURAÇÃO E ESTRUTURA DO TRABALHO}
\label{cap:configuracao}

Este capítulo tem por objetivo apresentar a estrutura do modelo do \ac{TCC} e descrever o processo de configuração do documento no editor online de documentos \textit{LaTex} Overleaf. A \autoref{sec:estruturaTC} apresenta a estrutura do trabalho de conclusão conforme a ABNT. A \autoref{sec:modeloTCC} descreve o modelo a ser utilizado e a sua estrutura. A \autoref{sec:primeiraConfiguracao} descreve os primeiros passos na configuração do trabalho. 

\section{ESTRUTURA DO TRABALHO DE CONCLUSÃO}
\label{sec:estruturaTC}
Conforme a norma ABNT NBR 14724:2011, a estrutura de teses, dissertações, \ac{TCC}, entre outros, compreende elementos obrigatórios e opcionais (e a sua respectiva ordem). Como o foco deste documento é \ac{TCC}, a Figura \ref{fig:estruturaTCC} apresenta os elementos obrigatórios e opcionais de um TCC. A ordem em que eles são apresentados deve ser mantido no documento. 

\begin{figure}[h]
	\caption{Estrutura do Trabalho de Conclusão para os cursos da Área da Informática}
	\centering 
	\label{fig:estruturaTCC}
	\UCSfigura {}{ 
	\includegraphics{figuras/estruturaTCC.png}
		}{
	\fonte{Adaptado de \citeonline{guiaUCS}.}
	}
\end{figure}    

Com o objetivo de simplificar o processo de confecção do \ac{TCC}, alguns elementos da estrutura não precisam constar na monografia submetida na primeira disciplina do \ac{TCC}. Tais elementos são mais apropriados para versão final da monografia na segunda e última disciplina de trabalho de conclusão de curso.  Os elementos são: 
\begin{itemize}
	\item Folha de aprovação
	\item Dedicatória
	\item Agradecimentos
	\item Epígrafe
	\item Resumo na língua estrangeira
\end{itemize}
Exceto a folha de aprovação, os demais elementos são opcionais e não precisam constar mesmo na monografia final.

É importante que o aluno utilize como base o guia de elaboração de trabalhos acadêmicos da UCS\cite{guiaUCS} para maiores detalhes.  

\section{MODELO DO TRABALHO DE CONCLUSÃO}
\label{sec:modeloTCC}

Para facilitar a formatação dos trabalhos de conclusão, este documento foi construído utilizando a ferramenta de edição de documentos \textit{LaTex} chamada \textit{Overleaf}\footnote{\url{http://overleaf.com}}. O \textit{Overleaf} é uma ferramenta \textit{online} que pode ser acessada por qualquer navegador de internet até mesmo de um \textit{smartphone}. Uma das vantagens desta ferramenta é a possibilidade de compartilhar o seu trabalho com o seu orientador. Assim, ele terá acesso imediato ao texto e realizar comentários acerca do trabalho. É necessário ter uma conta (grátis) em seu site para utilizá-lo. 

\subsection{Carregar o Modelo TCC} 
\label{subsec:loadTCC} 

Para começar o trabalho de edição, primeiramente é necessário o arquivo de projeto. O projeto nesta plataforma é um arquivo .zip. 

Após se conectar no site do \textit{Overleaf}, você deve realizar a carga do projeto. Clique em \textbf{Novo Projeto} e um menu de opções deve aparecer. Em seguida selecione a opção \textbf{Carregar Projeto}. Uma nova janela será aberta para você escolher ou arrastar um arquivo .zip. Utilize o projeto zipado disponibilizado pelo coordenador de \ac{TCC}. 

Altere o nome do projeto para facilitar a sua identificação. Uma sugestão é colocar TCC 1 ou 2 seguido pelo nome do aluno. Por exemplo, no caso do TCC 1 do aluno Carlos Heinz, o nome do projeto será TCC 1 - Carlos Heinz. 

\begin{tabular}{|l|c|} \hline
	DICA:  Carregue o projeto novamente para ter uma cópia de referência. 
                                                        \\\hline
\end{tabular}

\subsection{Estrutura do Projeto}
\label{sec:estruturaModelo} 

Ao abrir o projeto no Overleaf, a aba \textit{Project} (lateral esquerda) deve mostrar as pastas e arquivos, conforme ilustrado na \autoref{fig:modelo}:
\begin{alineas}
	\item 1-pre-textuais: pasta que contém os arquivos dos elementos pré-textuais: acrônimos, agradecimentos, folha de aprovação, dedicatória, epígrafe, resumos e lista de símbolos.
	\item 2-textuais: pasta que contém os arquivos dos elementos textuais (isto é, os capítulos da monografia --- da introdução à conclusão)
	\item 3-pos-textuais: pasta que contém os arquivos dos elementos pós-textuais: anexos, apêndices e a bibliografia.
	\item figuras: pasta que contém as figuras do trabalho.
	\item abntex2.cls: configuração da formatação (NÃO ALTERAR).
	\item TCC.tex: documento com as configurações do modelo. 
\end{alineas}

\begin{figure}[!ht]
	\caption{Aba \textit{Project} do modelo}
	\centering 
	\label{fig:modelo}
	\UCSfigura {}{ 
		\includegraphics{figuras/modelo.png}
		}{
		\fonte{O Autor (2021).}
	}
\end{figure}    

\subsection{Compartilhar o Projeto}
\label{sec:compartilhar} 

A versão grátis da ferramenta \textit{Overleaf} permite o compartilhamento do projeto com somente um usuário. No caso de um trabalho acadêmico, este compartilhamento poder ser feito com o orientador. 

Para compartilhar o projeto, deve-se estar editando o projeto. Clique no link \textbf{Compartilhar} no topo da página (lado direito). Uma janela será aberta para informar o e-mail de quem deve receber o compartilhamento. Em seguida, clique no botão compartilhar. 

\subsection{Atualizar o Projeto}
\label{sec:atualizar} 

O ambiente mostra na lateral direita uma versão compilada do documento para verificação. 

Para que cada alteração seja mostrada na versão compilada, deve-se clicar no botão Recompilar (acima do painel de visualização). 

No caso de alterar Para compartilhar o projeto, deve-se estar editando o projeto. Clique no link \textbf{Compartilhar} no topo da página (lado direito). Uma janela será aberta para informar o e-mail de quem deve receber o compartilhamento. Em seguida, clique no botão compartilhar. 


\section{PRIMEIRA CONFIGURAÇÃO}
\label{sec:primeiraConfiguracao} 

A primeira configuração tem por objetivo alimentar o projeto com as informações necessárias da monografia e atualizar a estrutura da monografia. 

\subsection{Identificações}

As informações de autoria, orientação e curso devem ser alteradas no arquivo TCC.tex, conforme ilustrado na \autoref{fig:info}. Informe os dados nos campos \texttt{titulo}, \texttt{autor}, \texttt{local}, \texttt{data} e \texttt{orientador}. No caso de não possuir um co-orientador, eliminar a linha do campo \texttt{coorientador} (linha 113).

A natureza do trabalho utilizada na folha de rosto e de aprovação é definida por meio do campo \texttt{preambulo}. Deve-se alterar somente o nome do curso (Ciência da Computação, Sistemas de Informação, Engenharia de Computação, ...). Para os alunos que são do campus da região dos Vinhedos, este campo pode incluir a identificação do campus: \textbf{Trabalho de Conclusão de Curso apresentado como requisito parcial à obtenção do título de Bacharel em Ciência da Computação na Área do Conhecimento de Ciências Exatas e Engenharias, Campus Universitário da Região dos Vinhedos, da Universidade de Caxias do Sul.  
}
\begin{figure}[!ht]
	\caption{Linhas do arquivo TCC.tex com as informações do trabalho}
	\centering 
	\label{fig:info}
	\UCSfigura {}{ 
		\includegraphics{figuras/info1.png}
		}{
		\fonte{O Autor (2021).}
	}
\end{figure}   

\subsection{Folha de Aprovação} 

Este elemento pré-textual deve ser inserido na monografia submetida na segunda disciplina de trabalho de conclusão. Duas informações devem ser preenchidas no arquivo aprovacao.tex na pasta 1-pre-textuais:

\begin{itemize}
    \item Data de Aprovação: deve ser preenchida com a data de apresentação de trabalho. Alterar o comando \verb!\large\textbf{Aprovado em 00/07/2021}!.
    \item Menbros da Banca: apesar do nome do orientador ser preenchido automaticamente, os membros da banca devem ser preenchidos manualmente. Os nomes dos membros da banca devem ser preenchidos com a respectiva titulação (Esp. --- Especialista, Me. ou Ma. --- Mestre, Dr. ou Dra. --- Doutor):
    \begingroup
    \fontsize{10pt}{12pt}\selectfont
    \begin{verbatim}
\assinatura{\ABNTEXchapterfont\large{Prof. Dra. Nome do Membro 1 da Banca} 
                                    \\ Universidade de Caxias do Sul - UCS}
\assinatura{\ABNTEXchapterfont\large{Prof. Dr. Nome do Membro 2 da Banca} 
                                    \\ Universidade de Caxias do Sul - UCS}
    \end{verbatim}
    \endgroup
\end{itemize}

\subsection{Elementos Opcionais} 

Alguns elementos do documento são opcionais ou não precisam necessariamente aparecer na monografia submetida na disciplina de Trabalho de Conclusão I \footnote{Verifique com a coordenação do seu curso sobre os itens obrigatórios que não precisam ser submetidos na respectiva disciplina}:


\begin{alineas}
	\item \textbf{Folha de aprovação}: este item é opcional na monografia submetida para a primeira disciplina de trabalho de conclusão, mas é obrigatória para aversão final na segunda e última disciplina. Para que este elemento não apareça, deve-se comentar as linhas no arquivo TCC.tex que o incluem:
	\begingroup
	\begin{verbatim}
    \newpage
    \include{1-pre-textuais/aprovacao}
	\end{verbatim}
	\endgroup
	\item \textbf{Dedicatória}: como este item é opcional, ele pode ser excluído do projeto. Nesse caso, deve-se apagar o arquivo dedicatoria.tex na pasta 1-pre-textuais e a linha no arquivo TCC.tex:
	\begingroup
	\begin{verbatim}
    \imprimirdedicatoria{1-pre-textuais/dedicatoria}
	\end{verbatim}
	\endgroup
	\item \textbf{Agradecimentos}: como este item é opcional, ele pode ser excluído do projeto. Nesse caso, deve-se apagar o arquivo agradecimentos.tex na pasta 1-pre-textuais e a linha no arquivo TCC.tex:
	\begingroup
	\begin{verbatim}
    \imprimiragradecimentos{1-pre-textuais/agradecimentos}
	\end{verbatim}
	\endgroup        
	\item \textbf{Epígrafe}: como este item é opcional, ele pode ser excluído do projeto. Nesse caso, deve-se apagar o arquivo epigrafe.tex na pasta 1-pre-textuais e a linha no arquivo TCC.tex:
	\begingroup
	\begin{verbatim}
    \include{1-pre-textuais/epigrafe}
	\end{verbatim}
	\endgroup   
	\item \textbf{Abstract}: o resumo na língua estrangeira é opcional. Por isso, ele pode ser excluído do projeto. Nesse caso, deve-se apagar o arquivo abstract.tex na pasta 1-pre-textuais e a linha no arquivo TCC.tex:.
	\begingroup
	\begin{verbatim}
    \include{1-pre-textuais/abstract}
	\end{verbatim}
	\endgroup   
	\item \textbf{Abreviaturas e Siglas}: como este item é opcional, ele pode ser excluído do projeto. No caso de exclusão, deve-se apagar o arquivo acronimos.tex na pasta 1-pre-textuais e a linha no arquivo TCC.tex:
	\begingroup
	\begin{verbatim}
    \include{1-pre-textuais/acronimos}
	\end{verbatim}
	\endgroup      
	\item \textbf{Símbolos}: como este item é opcional, ele pode ser excluído do projeto. No caso de exclusão, deve-se apagar o arquivo simbolos.tex na pasta 1-pre-textuais e a linha no arquivo TCC.tex:
	\begingroup
	\begin{verbatim}
    \include{1-pre-textuais/simbolos}
	\end{verbatim}
	\endgroup      
	\item \textbf{Apêndices}: se o trabalho não possui anexos, ele pode ser excluído do projeto. No caso de exclusão, deve-se apagar o arquivo anexos.tex na pasta 1-pre-textuais e a linha no arquivo TCC.tex:
	\begingroup
	\begin{verbatim}
    % ---
% Inicia os anexos
% ---
\begin{anexosenv}

	% ----------------------------------------------------------
	%\chapter{Anexo ...}
	%\label{anexo:nomea} 
	% ----------------------------------------------------------
	

	% ----------------------------------------------------------
	%\chapter{Anexo ...}
	%\label{anexo:nomeb} 
	% ----------------------------------------------------------

\end{anexosenv}

	\end{verbatim}
	\endgroup      
	\item \textbf{Anexos}: se o trabalho não possui anexos, ele pode ser excluído do projeto. No caso de exclusão, deve-se apagar o arquivo anexos.tex na pasta 1-pre-textuais e a linha no arquivo TCC.tex:
	\begingroup
	\begin{verbatim}
    % ---
% Inicia os anexos
% ---
\begin{anexosenv}

	% ----------------------------------------------------------
	%\chapter{Anexo ...}
	%\label{anexo:nomea} 
	% ----------------------------------------------------------
	

	% ----------------------------------------------------------
	%\chapter{Anexo ...}
	%\label{anexo:nomeb} 
	% ----------------------------------------------------------

\end{anexosenv}

	\end{verbatim}
	\endgroup
\end{alineas}

\subsection{Elementos Textuais}

Este projeto traz 6 capítulos, onde a introdução é o primeiro e o de conclusão o último. Caso o trabalho tenha menos capítulos, deve-se apagar a linha que o insere no documento no arquivo TCC.tex: 
 
\begingroup
\begin{verbatim}  
    \include{2-textuais/1-introducao}
    \chapter{CONFIGURAÇÃO E ESTRUTURA DO TRABALHO}
\label{cap:configuracao}

Este capítulo tem por objetivo apresentar a estrutura do modelo do \ac{TCC} e descrever o processo de configuração do documento no editor online de documentos \textit{LaTex} Overleaf. A \autoref{sec:estruturaTC} apresenta a estrutura do trabalho de conclusão conforme a ABNT. A \autoref{sec:modeloTCC} descreve o modelo a ser utilizado e a sua estrutura. A \autoref{sec:primeiraConfiguracao} descreve os primeiros passos na configuração do trabalho. 

\section{ESTRUTURA DO TRABALHO DE CONCLUSÃO}
\label{sec:estruturaTC}
Conforme a norma ABNT NBR 14724:2011, a estrutura de teses, dissertações, \ac{TCC}, entre outros, compreende elementos obrigatórios e opcionais (e a sua respectiva ordem). Como o foco deste documento é \ac{TCC}, a Figura \ref{fig:estruturaTCC} apresenta os elementos obrigatórios e opcionais de um TCC. A ordem em que eles são apresentados deve ser mantido no documento. 

\begin{figure}[h]
	\caption{Estrutura do Trabalho de Conclusão para os cursos da Área da Informática}
	\centering 
	\label{fig:estruturaTCC}
	\UCSfigura {}{ 
	\includegraphics{figuras/estruturaTCC.png}
		}{
	\fonte{Adaptado de \citeonline{guiaUCS}.}
	}
\end{figure}    

Com o objetivo de simplificar o processo de confecção do \ac{TCC}, alguns elementos da estrutura não precisam constar na monografia submetida na primeira disciplina do \ac{TCC}. Tais elementos são mais apropriados para versão final da monografia na segunda e última disciplina de trabalho de conclusão de curso.  Os elementos são: 
\begin{itemize}
	\item Folha de aprovação
	\item Dedicatória
	\item Agradecimentos
	\item Epígrafe
	\item Resumo na língua estrangeira
\end{itemize}
Exceto a folha de aprovação, os demais elementos são opcionais e não precisam constar mesmo na monografia final.

É importante que o aluno utilize como base o guia de elaboração de trabalhos acadêmicos da UCS\cite{guiaUCS} para maiores detalhes.  

\section{MODELO DO TRABALHO DE CONCLUSÃO}
\label{sec:modeloTCC}

Para facilitar a formatação dos trabalhos de conclusão, este documento foi construído utilizando a ferramenta de edição de documentos \textit{LaTex} chamada \textit{Overleaf}\footnote{\url{http://overleaf.com}}. O \textit{Overleaf} é uma ferramenta \textit{online} que pode ser acessada por qualquer navegador de internet até mesmo de um \textit{smartphone}. Uma das vantagens desta ferramenta é a possibilidade de compartilhar o seu trabalho com o seu orientador. Assim, ele terá acesso imediato ao texto e realizar comentários acerca do trabalho. É necessário ter uma conta (grátis) em seu site para utilizá-lo. 

\subsection{Carregar o Modelo TCC} 
\label{subsec:loadTCC} 

Para começar o trabalho de edição, primeiramente é necessário o arquivo de projeto. O projeto nesta plataforma é um arquivo .zip. 

Após se conectar no site do \textit{Overleaf}, você deve realizar a carga do projeto. Clique em \textbf{Novo Projeto} e um menu de opções deve aparecer. Em seguida selecione a opção \textbf{Carregar Projeto}. Uma nova janela será aberta para você escolher ou arrastar um arquivo .zip. Utilize o projeto zipado disponibilizado pelo coordenador de \ac{TCC}. 

Altere o nome do projeto para facilitar a sua identificação. Uma sugestão é colocar TCC 1 ou 2 seguido pelo nome do aluno. Por exemplo, no caso do TCC 1 do aluno Carlos Heinz, o nome do projeto será TCC 1 - Carlos Heinz. 

\begin{tabular}{|l|c|} \hline
	DICA:  Carregue o projeto novamente para ter uma cópia de referência. 
                                                        \\\hline
\end{tabular}

\subsection{Estrutura do Projeto}
\label{sec:estruturaModelo} 

Ao abrir o projeto no Overleaf, a aba \textit{Project} (lateral esquerda) deve mostrar as pastas e arquivos, conforme ilustrado na \autoref{fig:modelo}:
\begin{alineas}
	\item 1-pre-textuais: pasta que contém os arquivos dos elementos pré-textuais: acrônimos, agradecimentos, folha de aprovação, dedicatória, epígrafe, resumos e lista de símbolos.
	\item 2-textuais: pasta que contém os arquivos dos elementos textuais (isto é, os capítulos da monografia --- da introdução à conclusão)
	\item 3-pos-textuais: pasta que contém os arquivos dos elementos pós-textuais: anexos, apêndices e a bibliografia.
	\item figuras: pasta que contém as figuras do trabalho.
	\item abntex2.cls: configuração da formatação (NÃO ALTERAR).
	\item TCC.tex: documento com as configurações do modelo. 
\end{alineas}

\begin{figure}[!ht]
	\caption{Aba \textit{Project} do modelo}
	\centering 
	\label{fig:modelo}
	\UCSfigura {}{ 
		\includegraphics{figuras/modelo.png}
		}{
		\fonte{O Autor (2021).}
	}
\end{figure}    

\subsection{Compartilhar o Projeto}
\label{sec:compartilhar} 

A versão grátis da ferramenta \textit{Overleaf} permite o compartilhamento do projeto com somente um usuário. No caso de um trabalho acadêmico, este compartilhamento poder ser feito com o orientador. 

Para compartilhar o projeto, deve-se estar editando o projeto. Clique no link \textbf{Compartilhar} no topo da página (lado direito). Uma janela será aberta para informar o e-mail de quem deve receber o compartilhamento. Em seguida, clique no botão compartilhar. 

\subsection{Atualizar o Projeto}
\label{sec:atualizar} 

O ambiente mostra na lateral direita uma versão compilada do documento para verificação. 

Para que cada alteração seja mostrada na versão compilada, deve-se clicar no botão Recompilar (acima do painel de visualização). 

No caso de alterar Para compartilhar o projeto, deve-se estar editando o projeto. Clique no link \textbf{Compartilhar} no topo da página (lado direito). Uma janela será aberta para informar o e-mail de quem deve receber o compartilhamento. Em seguida, clique no botão compartilhar. 


\section{PRIMEIRA CONFIGURAÇÃO}
\label{sec:primeiraConfiguracao} 

A primeira configuração tem por objetivo alimentar o projeto com as informações necessárias da monografia e atualizar a estrutura da monografia. 

\subsection{Identificações}

As informações de autoria, orientação e curso devem ser alteradas no arquivo TCC.tex, conforme ilustrado na \autoref{fig:info}. Informe os dados nos campos \texttt{titulo}, \texttt{autor}, \texttt{local}, \texttt{data} e \texttt{orientador}. No caso de não possuir um co-orientador, eliminar a linha do campo \texttt{coorientador} (linha 113).

A natureza do trabalho utilizada na folha de rosto e de aprovação é definida por meio do campo \texttt{preambulo}. Deve-se alterar somente o nome do curso (Ciência da Computação, Sistemas de Informação, Engenharia de Computação, ...). Para os alunos que são do campus da região dos Vinhedos, este campo pode incluir a identificação do campus: \textbf{Trabalho de Conclusão de Curso apresentado como requisito parcial à obtenção do título de Bacharel em Ciência da Computação na Área do Conhecimento de Ciências Exatas e Engenharias, Campus Universitário da Região dos Vinhedos, da Universidade de Caxias do Sul.  
}
\begin{figure}[!ht]
	\caption{Linhas do arquivo TCC.tex com as informações do trabalho}
	\centering 
	\label{fig:info}
	\UCSfigura {}{ 
		\includegraphics{figuras/info1.png}
		}{
		\fonte{O Autor (2021).}
	}
\end{figure}   

\subsection{Folha de Aprovação} 

Este elemento pré-textual deve ser inserido na monografia submetida na segunda disciplina de trabalho de conclusão. Duas informações devem ser preenchidas no arquivo aprovacao.tex na pasta 1-pre-textuais:

\begin{itemize}
    \item Data de Aprovação: deve ser preenchida com a data de apresentação de trabalho. Alterar o comando \verb!\large\textbf{Aprovado em 00/07/2021}!.
    \item Menbros da Banca: apesar do nome do orientador ser preenchido automaticamente, os membros da banca devem ser preenchidos manualmente. Os nomes dos membros da banca devem ser preenchidos com a respectiva titulação (Esp. --- Especialista, Me. ou Ma. --- Mestre, Dr. ou Dra. --- Doutor):
    \begingroup
    \fontsize{10pt}{12pt}\selectfont
    \begin{verbatim}
\assinatura{\ABNTEXchapterfont\large{Prof. Dra. Nome do Membro 1 da Banca} 
                                    \\ Universidade de Caxias do Sul - UCS}
\assinatura{\ABNTEXchapterfont\large{Prof. Dr. Nome do Membro 2 da Banca} 
                                    \\ Universidade de Caxias do Sul - UCS}
    \end{verbatim}
    \endgroup
\end{itemize}

\subsection{Elementos Opcionais} 

Alguns elementos do documento são opcionais ou não precisam necessariamente aparecer na monografia submetida na disciplina de Trabalho de Conclusão I \footnote{Verifique com a coordenação do seu curso sobre os itens obrigatórios que não precisam ser submetidos na respectiva disciplina}:


\begin{alineas}
	\item \textbf{Folha de aprovação}: este item é opcional na monografia submetida para a primeira disciplina de trabalho de conclusão, mas é obrigatória para aversão final na segunda e última disciplina. Para que este elemento não apareça, deve-se comentar as linhas no arquivo TCC.tex que o incluem:
	\begingroup
	\begin{verbatim}
    \newpage
    \include{1-pre-textuais/aprovacao}
	\end{verbatim}
	\endgroup
	\item \textbf{Dedicatória}: como este item é opcional, ele pode ser excluído do projeto. Nesse caso, deve-se apagar o arquivo dedicatoria.tex na pasta 1-pre-textuais e a linha no arquivo TCC.tex:
	\begingroup
	\begin{verbatim}
    \imprimirdedicatoria{1-pre-textuais/dedicatoria}
	\end{verbatim}
	\endgroup
	\item \textbf{Agradecimentos}: como este item é opcional, ele pode ser excluído do projeto. Nesse caso, deve-se apagar o arquivo agradecimentos.tex na pasta 1-pre-textuais e a linha no arquivo TCC.tex:
	\begingroup
	\begin{verbatim}
    \imprimiragradecimentos{1-pre-textuais/agradecimentos}
	\end{verbatim}
	\endgroup        
	\item \textbf{Epígrafe}: como este item é opcional, ele pode ser excluído do projeto. Nesse caso, deve-se apagar o arquivo epigrafe.tex na pasta 1-pre-textuais e a linha no arquivo TCC.tex:
	\begingroup
	\begin{verbatim}
    \include{1-pre-textuais/epigrafe}
	\end{verbatim}
	\endgroup   
	\item \textbf{Abstract}: o resumo na língua estrangeira é opcional. Por isso, ele pode ser excluído do projeto. Nesse caso, deve-se apagar o arquivo abstract.tex na pasta 1-pre-textuais e a linha no arquivo TCC.tex:.
	\begingroup
	\begin{verbatim}
    \include{1-pre-textuais/abstract}
	\end{verbatim}
	\endgroup   
	\item \textbf{Abreviaturas e Siglas}: como este item é opcional, ele pode ser excluído do projeto. No caso de exclusão, deve-se apagar o arquivo acronimos.tex na pasta 1-pre-textuais e a linha no arquivo TCC.tex:
	\begingroup
	\begin{verbatim}
    \include{1-pre-textuais/acronimos}
	\end{verbatim}
	\endgroup      
	\item \textbf{Símbolos}: como este item é opcional, ele pode ser excluído do projeto. No caso de exclusão, deve-se apagar o arquivo simbolos.tex na pasta 1-pre-textuais e a linha no arquivo TCC.tex:
	\begingroup
	\begin{verbatim}
    \include{1-pre-textuais/simbolos}
	\end{verbatim}
	\endgroup      
	\item \textbf{Apêndices}: se o trabalho não possui anexos, ele pode ser excluído do projeto. No caso de exclusão, deve-se apagar o arquivo anexos.tex na pasta 1-pre-textuais e a linha no arquivo TCC.tex:
	\begingroup
	\begin{verbatim}
    \include{3-pos-textuais/anexos}
	\end{verbatim}
	\endgroup      
	\item \textbf{Anexos}: se o trabalho não possui anexos, ele pode ser excluído do projeto. No caso de exclusão, deve-se apagar o arquivo anexos.tex na pasta 1-pre-textuais e a linha no arquivo TCC.tex:
	\begingroup
	\begin{verbatim}
    \include{3-pos-textuais/anexos}
	\end{verbatim}
	\endgroup
\end{alineas}

\subsection{Elementos Textuais}

Este projeto traz 6 capítulos, onde a introdução é o primeiro e o de conclusão o último. Caso o trabalho tenha menos capítulos, deve-se apagar a linha que o insere no documento no arquivo TCC.tex: 
 
\begingroup
\begin{verbatim}  
    \include{2-textuais/1-introducao}
    \include{2-textuais/2-Configuracao}
    \include{2-textuais/3-Comandos}
    \include{2-textuais/4-Referencias}
    \include{2-textuais/5-Dicas}
    \include{2-textuais/6-Conclusoes}
\end{verbatim}
\endgroup

A numeração no nome do arquivo é para garantir uma ordenação dos arquivos. Caso o nome do arquivo seja alterado, deve-se manter a numeração para garantir a ordem dos capítulos na visualização. 


    \chapter{COMANDOS LATEX}
\label{cap:comandos}

Este capítulo tem por objetivo descrever os comandos \textit{LaTex} para a confecção da monografia. A \autoref{sec:divisao} descreve como é a divisão de documentos em conformidade com a ABNT NBR 6024 \cite{NBR6024:2012}. 

Consulte os manuais do \abnTeX\
\cite{abntex2classe,abntex2cite,abntex2cite-alf} para maiores detalhes sobre o uso deste template. Exemplos e informações adicionais de uso de \abnTeX\ e de \LaTeX podem ser encontrados em \citeonline{abntex2modelo}.

\section{DIVISÃO DE DOCUMENTOS}
\label{sec:divisao}

A principal divisão de uma monografia é chamada de seção primária (conhecida por capítulo)\cite{NBR6024:2012}. Para criar um capítulo, deve-se utilizar o comando \verb!\chapter{}!. Por exemplo, este capítulo foi definido conforme o comando:

\begin{verbatim}
    \chapter{COMANDOS LATEX}
\end{verbatim}

Um capítulo é dividido em seções, chamadas de secundárias. Uma seção secundária é criada utilizando o comando

\begin{verbatim}
    \section{<nome da seção>} 
\end{verbatim}

Da mesma maneira que um capítulo é referenciado, qualquer seção (não importa o seu nível) pode ser referenciada utilizando o \verb!autoref{}!

% ---
\subsection{Seção Terciária}
\label{sec:secaoTerciaria}
% ---

 Uma seção secundária pode ser dividida em mais seções, chamadas de terciária. Esta é a \autoref{sec:secaoTerciaria}. Uma seção terciária é criada utilizando o comando 

\begin{verbatim}
    \subsection{<nome da subseção>} 
\end{verbatim}

\subsubsection{Seção Quaternária}
\label{sec:secaoQuaternaria}

A subdivisão do texto a partir de uma seção terciária é chamada de seção quaternária. Uma seção quaternária é criada utilizando o comando 

\begin{verbatim}
    \subsubsection{<nome da subseção>} 
\end{verbatim}

Não se recomenda a divisão do texto a partir de uma seção quaternária. 

\section{REFERENCIAR SEÇÕES, CAPÍTULOS E OUTROS ELEMENTOS}

Em um texto técnico, é muito comum que capítulos ou seções sejam referenciados. Além de existir uma formatação própria para realizar o referenciamento, deve-se garantir que a numeração esteja sempre correta (a numeração das divisões do documento podem mudar durante a sua construção). 

De modo a eliminar qualquer dúvida sobre os termos a serem utilizados ao referenciar, somente as palavras apêndice, anexo, capítulo e seção devem ser utilizadas para referenciar qualquer divisão do documento. Não existe lugar para o uso de subseção ou troca do termo capítulo por seção, ou vice-versa. Além disso, como o termo é seguido por um número identificador, a primeira letra deve estar em maiúscula. 

Para que a divisão do documento seja referenciada conforme as normas, deve-se adicionar o comando de definição de rótulo \verb!\label{}!. Por exemplo, este capítulo foi definido conforme os comandos:

\begin{verbatim}
     \chapter{COMANDOS LATEX}
     \label{cap:comandos}
\end{verbatim}

Para referenciar uma divisão, utiliza-se o comando \verb!autoref{}! com o rótulo designado. Além de realizar uma referência ao número da divisão, o comando produz o termo apropriado (depende do tipo de título). 

No caso deste capítulo, o comando \verb!\autoref{cap:comandos}! produzirá o texto \autoref{cap:comandos}. Nota-se que o capítulo deve possuir um rótulo definido como "cap:estrutura". No caso de seções, o texto produzido é "\autoref{sec:secaoTerciaria}." ou "\autoref{sec:secaoQuaternaria}." Em todos os casos, o termo é gerado com a primeira letra maiúscula conforme a formatação da  \ac{ABNT}. 

No caso de múltiplas seções serem referenciadas, deve-se utilizar o termo da divisão seguido pelo comando \verb!\ref{}! para a inserção do número. Por exemplo,

"As Seções \ref{sec:enum} e \ref{sec:divisao} tratam..." 

"Os Capítulos \ref{cap:configuracao} e \ref{cap:consideracoes} apresentam..."

É importante que o termo tenha a primeira letra em maiúsculo. 


\section{ILUSTRAÇÕES}

As ilustrações incluem figuras, tabelas, quadros e algoritmos. Quando utilizadas, estas ilustrações produzem uma lista na parte pré-textual do documento. 

\subsection{Figuras}

As ilustrações devem sempre ter numeração contínua e única em todo o documento:

\begin{citacao}
	Qualquer que seja o tipo de ilustração, sua identificação aparece na parte
	superior, precedida da palavra designativa (desenho, esquema, fluxograma,
	fotografia, gráfico, mapa, organograma, planta, quadro, retrato, figura,
	imagem, entre outros), seguida de seu número de ordem de ocorrência no texto,
	em algarismos arábicos, travessão e do respectivo título. Após a ilustração, na
	parte inferior, indicar a fonte consultada (elemento obrigatório, mesmo que
	seja produção do próprio autor), legenda, notas e outras informações
	necessárias à sua compreensão (se houver). A ilustração deve ser citada no
	texto e inserida o mais próximo possível do trecho a que se
	refere \cite{NBR14724:2011}.
\end{citacao}

A \autoref{fig:exemploFigura} mostra um exemplo de figura, com os principais elementos descritos pela norma. A referência a uma figura deve utilizar o comando \verb!autoref{}!. O comando \texttt{UCSfigura} garante o alinhamento da fonte no lado esquerdo da figura.  

    \begin{figure}[!ht]
    	\caption{Exemplo de figura}
    	\centering 
    	\label{fig:exemploFigura}
    	\UCSfigura {}{ 
    		\includegraphics{figuras/estruturaTCC.png}
    		}{
    		\fonte{Adaptado de \citeonline{guiaUCS}.}
    	}
    \end{figure}   

Algumas das figuras podem ser formadas por várias figuras e as mesmas referenciadas no texto. O pacote \texttt{subfigure}  permite inserir e identificar cada uma destas figuras. A \autoref{fig:estrutura} mostra um exemplo de como colocar duas figuras lado a lado com rótulos próprios para referenciamento no texto. A \autoref{fig:estuturaABNT1} aponta para a figura da esquerda e \autoref{fig:estuturaTCC1} a da direita. 


\begin{figure}[!ht]
	\centering
	\caption{Estruturas de trabalhos pela ABNT e o adotado pela Área da Informática da UCS.}
	\label{fig:estrutura} 
	\begin{subfigure}{0.45\textwidth}
		\centering
		\includegraphics[width = \textwidth]{figuras/estruturaABNT.png}
		\caption{Estrutura de Documentos conforme a ABNT}
		\label{fig:estuturaABNT1}
	\end{subfigure}
	\begin{subfigure}{0.45\textwidth}
		\centering
		\includegraphics[width = \textwidth]{figuras/estruturaTCC.png}
		\caption{Estrutura do TCC }
		\label{fig:estuturaTCC1}
	\end{subfigure}
			
	\fonte{O Autor (2021).}
		
\end{figure}

O \autorefanexo{anex:exemploFiguras} apresenta mais exemplos de figuras que utilizam o \texttt{subfigure}. 

No caso de imagens retiradas da internet, não é necessário adicionar a URL do site na lista de referências do trabalho. Como o objetivo é somente ilustrar, pode-se referenciar o website. A \autoref{fig:exemploImagemInternet} mostra um exemplo de citação da fonte de internet. 

\begin{figure}[!ht]
	\caption{Exemplo de figura com fonte da internet.}
	\centering 
	\label{fig:exemploImagemInternet}
	\UCSfigura {}{ 
		\includegraphics[width = 0.45\textwidth]{figuras/biblio-central.jpg}
		}{ 	\fonte{Disponível em: <https://www.ucs.br/site/biblioteca/biblioteca-central/>. Acesso em 21 mai. 2021} 
		}
\end{figure} 

% ---
\subsection{Tabelas}
% ---
As tabelas apresentam informações numéricas tratadas
estatisticamente \cite{ibge1993}. A estrutura da tabela é constituída de traços (retas perpendiculares) e delimitada, na parte superior e na parte inferior por traços horizontais paralelos. Não delimitar (fechar) por traços verticais os extremos da tabela à direita e à esquerda. A \autoref{tab:ibge} é um exemplo de tabela conforme o padrão IBGE requerido pelas normas da ABNT para documentos técnicos e acadêmicos. Além da fonte na parte inferior da tabela, o exemplo traz como se insere uma nota e anotações, quando necessários. O comando \texttt{IBGEtab} garante o alinhamento da fonte no lado esquerdo da tabela.  

\begin{table}[!ht]
	\caption{\label{tab:ibge} Um Exemplo de tabela alinhada que pode ser longa ou curta}%
	\IBGEtab{}{
		\begin{tabular}{lccc}
			\toprule
			\textbf{Pessoa} & \textbf{Idade} & \textbf{Peso} & \textbf{Altura} \\
			\midrule \midrule
			Marcos          & 26             & 68            & 178             \\
			Ivone           & 22             & 57            & 162             \\ 
			...             & ...            & ...           & ...             \\
			Sueli           & 40             & 65            & 153             \\ 
			\bottomrule
		\end{tabular}
		}{
		\fonte{O Autor (2021).}
		\nota{esta é uma nota, que diz que os dados são baseados na
		regressão linear.}
		\nota[Anotações]{uma anotação adicional, seguida de várias outras.}
	}
\end{table}
    
O \autoref{cap:exemploTabelas} apresenta mais exemplos de tabelas, incluindo uma tabela que ultrapassa uma página (formatação própria). 

% ---
\subsection{Quadros}
% ---

A formatação do quadro é similar à tabela, mas deve ter suas laterais fechadas e conter as linhas horizontais. O \autoref{quad:exemplo} mostra um exemplo do uso de quadro no texto. O comando \texttt{UCSQuadro} garante o alinhamento da fonte no lado esquerdo do quadro.   

\begin{quadro}[!ht]
	\caption{\label{quad:exemplo} Exemplo de quadro}
	\UCSQuadro{}{%
	\begin{tabular}{|c|l|c|}
		\hline
		\textbf{Norma} & \textbf{Título} & \textbf{Ano de Publicação} \\
		\hline
		NBR 14724  & Trabalho Acadêmico    & 2011  \\ \hline
		NBR 10520   & Citações   & 2002   \\ \hline
		NBR 6023  & Referências & 2018 (Versão Corrigida 2:2020)  \\ \hline
		NBR 6027   & Sumário    & 2012  \\ \hline
		NBR 6028  & Resumo e Abstract     & 2003   \\ \hline
		NBR 6024  & Numeração progressiva das seções    & 2012 \\ \hline
		NBR 6034  & Índice    & 2004   \\ \hline
	\end{tabular}
	}{
	\fonte{O Autor (2021).
	}
}\end{quadro}

O \autoref{cap:exemploQuadros} apresenta mais exemplos de quadros, incluindo um quadro que ultrapassa uma página (formatação própria). 

\subsection{Algoritmos/Códigos Fontes}    
    
Não existe uma norma que padronize a formatação de algoritmos ou código fontes em trabalhos técnicos. Neste modelo, utiliza-se o pacote \textit{Listings} \footnote{https://ctan.org/pkg/listings} para a formatação. 

Para facilitar a utilização deste recurso, a linguagem padrão de formatação é C, conforme consta no arquivo TCC.tex e ilustrado na  \ref{fig:linguagem}. Caso não seja C a linguagem mais utilizada, substitua pelo nome da linguagem a ser utilizada conforme o padrão definido na documentação do pacote \textit{Listings}.

\begin{figure}[!ht]
\centering 
\caption{Configuração do pacote \textit{Listings}}
\label{fig:linguagem}
\UCSfigura{}{%
    \includegraphics[scale=0.70]{figuras/linguagem.png}
}{%
 \fonte{O Autor (2021).}%
 }
\end{figure}

Um exemplo é apresentado no Algoritmo \ref{codigo:exemplo1}. Nota que para a configuração deste algoritmo, foram utilizadas somente 2 campos: \textit{caption} e \textit{label}. 

\begin{lstlisting}[ caption={Exemplo de código em C},
                    label={codigo:exemplo1}]
#include <stdio.h>
int main() {    
    int numero1, numero2, soma;
    
    printf("Digite 2 numeros: ");
    scanf("%d %d", &numero1, &numero2);

    // Calcular a soma
    soma = numero1 + numero2;      
    
    printf("%d + %d = %d", numero1, numero2, soma);
    return 0;
}
\end{lstlisting}
\fonte{O Autor (2021)}

No caso de precisar adicionar um código que não seja da linguagem padrão, pode-se fazer utilizando o campo \textit{language} ( \verb!\begin{lstlisting}[language=Python,...]!). Um exemplo de um código na linguagem Python é mostrado no Algoritmo \ref{codigo:exemplo2}).

\begin{lstlisting}[language=Python,
                    caption={Exemplo de código em Python},
                    label={codigo:exemplo2}]
n = int(input("Digite o valor de n: "))
fatorial = 1
while (n > 0):
    fatorial = fatorial * n
    n -= 1
print(fatorial)
\end{lstlisting}
\fonte{O Autor (2021)}

\section{SIGLAS}
\label{sec:siglas}

Para utilizar siglas, primeiro defina a sigla no arquivo "acronimos.tex"~ na pasta "1-pre-textuais"~com o comando 

 \begingroup
\fontsize{10pt}{12pt}\selectfont
   \verb!\acro{PC}{Personal Computer}!
\endgroup
    
O comando deve ser colocado no arquivo de tal maneira que mantenha a ordem alfabética. O \textit{LaTex} garante somente se ela deve ser mostrada ou não na lista.

Para referenciar a sigla, deve-se utilizar o comando:

\begingroup
\fontsize{10pt}{12pt}\selectfont
\verb!\ac{PC}! \newline
\endgroup

A primeira vez que o comando é usado para uma determinada sigla, aparece o significado por extenso da sigla com a sua abreviação em seguida, desta maneira --- \ac{PC}. A partir da segunda vez que o comando para uma determinada sigla é usado, aparace apenas a sigla --- \ac{PC}. Assim, o autor nunca terá que se preocupar com a regra de sempre colocar a sigla por extenso na primeira vez. 

No caso de utilizar o plural da sigla, utilize-o com a letra s após a referência, como em \acp{PC}. Para referenciar a sigla no plural, deve-se utilizar o comando:

\begingroup
\fontsize{10pt}{12pt}\selectfont
\verb!\acp{PC}! \newline
\endgroup

É muito comum ver em trabalhos o uso de 's após a sigla. Este tipo de construção é utilizada para o possessivo da língua inglesa. 

\textbf{Não use siglas nos títulos e evite fortemente usar no Resumo. }

% ---
\section{ENUMERAÇÕES: ALÍNEAS e SUB-ALÍNEAS}
\label{sec:enum} 
% ---

Quando for necessário enumerar os diversos assuntos de uma seção que não possua título, esta deve ser subdividida em alíneas \cite[4.2]{NBR6024:2012}:

\begin{alineas}
  \item os diversos assuntos que não possuam título próprio, dentro de uma mesma
  seção, devem ser subdivididos em alíneas; 
  
  \item o texto que antecede as alíneas termina em dois pontos;
  \item as alíneas devem ser indicadas alfabeticamente, em letra minúscula,
  seguida de parêntese. Utilizam-se letras dobradas, quando esgotadas as
  letras do alfabeto;

  \item as letras indicativas das alíneas devem apresentar recuo em relação à
  margem esquerda;

  \item o texto da alínea deve começar por letra minúscula e terminar em
  ponto-e-vírgula, exceto a última alínea que termina em ponto final;

  \item o texto da alínea deve terminar em dois pontos, se houver subalínea;

  \item a segunda e as seguintes linhas do texto da alínea começa sob a
  primeira letra do texto da própria alínea;
  
  \item subalíneas \cite[4.3]{NBR6024:2012} devem ser conforme as alíneas a
  seguir:

  \begin{alineas}
     \item as subalíneas devem começar por travessão seguido de espaço;

     \item as subalíneas devem apresentar recuo em relação à alínea;

     \item o texto da subalínea deve começar por letra minúscula e terminar em
     ponto-e-vírgula. A última subalínea deve terminar em ponto final, se não
     houver alínea subsequente;

     \item a segunda e as seguintes linhas do texto da subalínea começam sob a
     primeira letra do texto da própria subalínea.
  \end{alineas}
  
  \item no \abnTeX\ estão disponíveis os ambientes \texttt{incisos} e
  \texttt{subalineas}, que em suma são o mesmo que se criar outro nível de
  \texttt{alineas}, como nos exemplos a seguir:
  
  \begin{incisos}
    \item \textit{Um novo inciso em itálico};
  \end{incisos}
  
  \item Alínea em \textbf{negrito}:
  
  \begin{subalineas}
    \item \textit{Uma subalínea em itálico};
    \item \underline{\textit{Uma subalínea em itálico e sublinhado}}; 
  \end{subalineas}
  
  \item Última alínea com \emph{ênfase}.
  
\end{alineas}

É possível modificar a enumeração para utilizar outras sequências. Deve-se colocar o novo símbolo entre colchetes após o comando \verb=\item[]=. Um exemplo de enumeração alterando a sequência pode ser definido da seguinte forma:

\begin{alineas}
	\item[1.] primeiro item
	\item[2.] segundo item
	\begin{alineas}
		\item[a)] primeiro subalínea 
		\item[b)] segundo subalínea
	\end{alineas}
	\item[3.] terceiro item
\end{alineas}

Existem diversos símbolos pré-definidos que podem ser utilizados para a enumeração:

\begin{alineas}
	\item[$\times$] primeiro item
	\item[$\bullet$] segundo item
	\item[$\diamond$] terceiro item
	\item[$\cdot$] quarto item
	\item[$\ast$] quinto item
	\item[$\circ$] sexto item
	\item[$-$] sétimo item
\end{alineas}

\section{CITAÇÕES} 

As citações são menções de informações extraídas de outras fontes \cite{NBR10520:2002}. Elas podem ser uma transcrição textual de parte da obra (citação direta) ou um texto que transcreve a informação com as palavras do autor do trabalho (citação indireta). 
No caso de citações indiretas, existem duas formas de realizar a citação:

\begin{itemize}
	\item \textbf{Início ou meio da frase}: deve-se citar o autor com a primeira letra em maiúsculo e colocar em seguida, entre parênteses o ano. Para esta construção, utiliza-se o comando \verb=\citeonline{}=. Por exemplo,  \citeonline{Acevedo2013}. No caso de querer adicionar o intervalo de páginas, o comando é  \verb=\citeonline[p. 10--15]{key}=. Por exemplo, \citeonline[pp. 21--30]{Acevedo2013}. No caso de somente uma página, o comando é  \verb=\citeonline[p. 10]{key}=. Por exemplo, \citeonline[pp. 21]{Acevedo2013}.
	          
	\item \textbf{Final da frase}: deve-se citar entre parênteses o autor em letras maiúsculas e colocar em seguida, entre parênteses o ano. Para esta construção, utiliza-se o comando \verb=\cite{}=. Por exemplo, \cite{Acevedo2013}. No caso de querer adicionar o intervalo de páginas, o comando é \verb=\cite[p. 10--15]{key}=. Por exemplo, \cite[pp. 21--30]{Acevedo2013}. No caso de somente uma página, o comando é  \verb=\cite[p. 10]{key}=. Por exemplo, \cite[p. 21]{Acevedo2013}.
\end{itemize}

Para realizar a citação direta de um texto longo (mais de 3 linhas), deve-se utilizar o seguinte comando: \verb=\begin{citacao} trecho transcrito \end{citacao}=. Por exemplo, 
\begin{citacao}
	As citações diretas, no texto, com mais de três linhas, devem ser destacadas com recuo de 4 cm da margem esquerda, com letra menor que a do texto utilizado e sem as aspas. No caso de documentos datilografados, deve-se observar apenas
	o recuo \cite[p. 2]{NBR10520:2002}.
\end{citacao}

No caso de não ter acesso à publicação original, é possível realizar a citação de uma citação, ou seja, a transcrição direta ou indireta de um texto ao qual não se teve acesso ao original. Existe uma formatação própria para realizar este tipo de citação utilizando o termo apud ("citado por"). A citação é composta pelo nome do autor original (ao qual não se teve acesso à obra e não deve constar nas referências do trabalho), o termo apud e o nome do autor da obra consultada (a qual deve constar nas referências do trabalho). Por exemplo, no Guia de Trabalhos Acadêmicos da UCS (\cite{guiaUCS}) existe a menção da norma 12225 da ABNT. A sua citação pode ser realizada da seguinte maneira:

\begingroup
\begin{adjustwidth}{2cm}{2cm}
    \noindent
	As informações devem ser impressas conforme a norma ABNT NBR 12225:2004 (\citeauthor{nbr12225:2004}, \citeyear{nbr12225:2004}\footnote{ASSOCIAÇÃO BRASILEIRA DE NORMAS TÉCNICAS.\textbf{NBR 10520}: Informação e documentação — lombada — apresentação. Rio de Janeiro, 2004. 3 p.} apud \citeauthor{guiaUCS}, \citeyear{guiaUCS}). 
	
	\centering ou
	
	\justifying
	\noindent\citeauthor{nbr12225:2004}\footnote{ASSOCIAÇÃO BRASILEIRA DE NORMAS TÉCNICAS.\textbf{NBR 10520}: Informação e documentação — lombada — apresentação. Rio de Janeiro, 2004. 3 p.} ( \citeyear{nbr12225:2004} apud \citeauthor{guiaUCS}, \citeyear{guiaUCS}) define as normas de como as informações devem ser impressas na lombada. 
\end{adjustwidth}
\endgroup

Em ambos os exemplos, o texto da referência original deve ser colocada no rodapé. De modo a reduzir a digitação, os dados da referência original podem ser colocados no arquivo de bibliografias (referencias.bib), mas utilizando o identificador @hidden (para que não faça parte da lista de referências):

\begingroup
\fontsize{10pt}{12pt}\selectfont
\begin{verbatim}
      @hidden{NBR12225:2004, 
          Address = {Rio de Janeiro},  
          Organization = {ASSOCIAÇÃO BRASILEIRA DE NORMAS TÉCNICAS}, 
          Pages = 3,  
          Subtitle = {Informação e documentação - lombada - apresentação},
          Title = {{NBR} 12225}, 
          Year = 2004
       }  
\end{verbatim}
\endgroup

% ---
\section{EXPRESSÕES MATEMÁTICAS}
\label{sec:expressoes}
% ---

As expressões matemáticas devem ser escritas utilizando ambiente \texttt{equation} para serem numeradas (para que sejam referenciadas):

\begin{equation}\label{eq:exemplo1}
\hat{c} = \underset{c}{\arg\max}  P(\textbf{x}|c).
\end{equation}

A referência a uma equação deve utilizar o comando \verb!autoref{}!. A \autoref{eq:exemplo1} é um exemplo de referência. 

No caso de uma expressão não necessitar ser numerada, é possível usar colchetes para indicar o início de uma expressão:

\[
\left|\sum_{i=1}^n a_ib_i\right|
\le
\left(\sum_{i=1}^n a_i^2\right)^{1/2}
\left(\sum_{i=1}^n b_i^2\right)^{1/2}
\]

Se a expressão matemática puder estar na mesma linha do texto, pode-se escrever expressões matemáticas entre cifrões (\$), como em $ \lim_{x \to \infty} \exp(-x) = 0 $.

Consulte mais informações sobre expressões matemáticas em
\url{https://github.com/abntex/abntex2/wiki/Referencias}.

\section{FORMATAÇÃO DO FONTE}

Para uma formatação do \textit{LaTex}, pode-se utilizar o formatador online do C. Albert Thompson  \footnote{https://c.albert-thompson.com/latex-pretty/}. Este site realiza a indentação do código-fonte \textit{LaTex}. 


    \include{2-textuais/4-Referencias}
    \chapter{DICAS GERAIS}
\label{cap:dicas}
	
Este capítulo tem por objetivo fornecer orientações para a construção da monografia. A \autoref{sec:capitulos} descreve algumas orientações sobre a construção dos capítulos. 

\section{CAPÍTULOS} 
\label{sec:capitulos} 

Todo trabalho de conclusão possui os elementos textuais Introdução e Conclusões. A questão que resta é como estruturar o trabalho entre estes dois capítulos. Nesta seção, serão discutidos algumas orientações sobre como um capítulo pode ser construído. 

Em uma monografia, espera-se que o texto esteja organizado de tal forma que o leitor consiga acompanhar e compreender o que está sendo feito e porquê. Assim, geralmente o desenvolvimento está distribuído em capítulos que tratem do referencial teórico, proposta e experimentos/resultados/discussão. Este último aparece na monografia da segunda disciplina de trabalho de conclusão. 

Em algumas situações, o referencial teórico pode aparecer em dois capítulos devido a sua complexidade ou amplitude. Por exemplo, pode ser que um capítulo forneça uma visão geral da área e as respectivas definições enquanto o o outro trate das características do problema específico do trabalho. O objetivo de dividir em mais de um capítulo é permitir um texto mais conciso e não maçante para o leitor. O texto deve sempre considerar que a compreensão do texto pode ser gradual e acumulativa. 

Um capítulo é formado por texto que apresenta algum tipo de relação. Por exemplo, em um capítulo de referencial teórico, espera-se que o seu conteúdo seja formado por definições e relações com outros trabalhos afins. Em um capítulo que apresente uma proposta de solução, o texto deve tratar somente dos elementos desta solução. Para facilitar a leitura, inicie o capítulo explicando o motivo ou objetivo do mesmo. Além disso, pode-se descrever a estrutura dos capítulos apresentando cada uma das seções secundárias (como é feito no início deste capítulo). Assim, evita-se o uso de frases que apresentam a próxima seção. 

Uma seção que é geralmente incluída no final dos capítulos é a de considerações finais. Esta seção tem por objetivo desenhar conclusões sobre o que foi tratado no texto. Ela auxilia na leitura, pois traz os principais pontos do capítulo que serão considerados no(s) próximo(s) capítulo(s). Mas um erro que muitos cometem nesta seção é a de introduzir o que será tratado no próximo capítulo. É necessário lembrar que a seção de estrutura do trabalho na introdução e o texto inicial do capítulo já realizam este papel de explicar o que cada capítulo traz. 

\section{PARÁGRAFOS}

O parágrafo é a unidade de discurso de um texto \cite{martins2019}. É a divisão do texto em partes menores, que determinam um enfoque. Portanto, uma seção deve ser composta por parágrafos e não frases soltas. 

O parágrafo possui partes distintas\cite{martins2019}. O tópico frasal é a ideia principal do parágrafo. Exprime a ideia principal do parágrafo. Em seguida, frases que dão suporte ou explicam em maiores detalhes o tópico frasal são apresentadas. É possível ainda ter uma conclusão que resuma o parágrafo ou exponha o ponto de maior interesse ou um elemento relacionador que estabeleça o encadeamento entre os parágrafos. Toda vez que trocar o enfoque do parágrafo, deve-se criar um parágrafo. 

Como a primeira frase de um parágrafo estabelece a principal ideia de cada parágrafo, deve ser possível compreender o texto da seção a partir de cada frase. As demais frases de cada parágrafo devem somente dar mais detalhes sobre a ideia principal. Por isso, um parágrafo não deve ser iniciado apresentando uma ilustração.  

\section{APÊNDICES X ANEXOS}

Uma confusão muito comum em trabalhos é o uso de apêndices e anexos. O apêndice "é o texto ou documento elaborado pelo próprio autor, de modo a complementar o texto principal e é apresentado no final do trabalho" \cite{guiaUCS}. O anexo é "destinado à inclusão de materiais não 
elaborados pelo próprio autor, como cópias de artigos, manuais, fôlderes, balancetes, etc., visando a dar suporte à argumentação, fundamentação, ilustração ou comprovação" \cite{guiaUCS}. 

O apêndice e o anexo são elementos opcionais. Somente os títulos (seções primárias) destes elementos devem constar no sumário. Caso tais elementos possuírem ilustrações, as mesmas não devem constar nas respectivas listas da monografia. 

\section{ILUSTRAÇÕES}

Uma boa forma de explicar algo é usando ilustrações (desenhos, esquemas, fluxogramas, fotografias, gráficos, mapas, organogramas, plantas, quadros, retratos e outros). Uma ilustração por si só não faz sentido em um texto, ou seja, a ilustração deve estar sempre relacionada a um tópico do texto. 

Toda ilustração deve ser citada no texto e explicada. O texto deve indicar para o leitor o momento certo para olhar ou analisar a ilustração. Por isso, em algumas situações, deve-se explicar a ilustração. 

Se você entender que as ilustrações estão para auxiliar na compreensão do texto, isso vai facilitar o que deve ser escrito no texto ao referenciar a ilustração. 

\section{EQUAÇÕES}

A primeira consideração na utilização de equações é que elas são parte do texto. Isto é, elas podem ser o meio ou o fim de uma frase e, portanto, deve ter a pontuação definida. Por exemplo, pode-se utilizar a fórmula da relatividade $E = mc^2$ como parte da frase (deve ser limitada por cifrões. Ou ainda, pode-se utilizá-las de forma separada (em nova linha) para dar destaque e também numerá-la: 
\begin{citacao}
Este método normaliza cada variável utilizando a equação
\begin{equation}\label{eq:exemplo2}
Z=\frac{X-\mu}{\sigma},
\end{equation}

onde $\mu$ e $\sigma$ são a média e o desvio padrão estimados a partir do conjunto de vetores de características denotado por $X$.

Com base no teorema descrito, a \autoref{eq:exemplo2} pode ser reescrita...

\end{citacao}

Nota-se que após a equação, utiliza-se a vírgula como continuação da frase. É importante que todas as variáveis da equação sejam explicadas antes ou depois dela. Para que o texto após a equação comece na margem esquerda, não deve haver uma linha em branco após a equação.

Deve-se manter uma utilização consistente de variáveis ao longo do texto. NUNCA reutilize variáveis para definições diferentes. 





    \include{2-textuais/6-Conclusoes}
\end{verbatim}
\endgroup

A numeração no nome do arquivo é para garantir uma ordenação dos arquivos. Caso o nome do arquivo seja alterado, deve-se manter a numeração para garantir a ordem dos capítulos na visualização. 


    \chapter{COMANDOS LATEX}
\label{cap:comandos}

Este capítulo tem por objetivo descrever os comandos \textit{LaTex} para a confecção da monografia. A \autoref{sec:divisao} descreve como é a divisão de documentos em conformidade com a ABNT NBR 6024 \cite{NBR6024:2012}. 

Consulte os manuais do \abnTeX\
\cite{abntex2classe,abntex2cite,abntex2cite-alf} para maiores detalhes sobre o uso deste template. Exemplos e informações adicionais de uso de \abnTeX\ e de \LaTeX podem ser encontrados em \citeonline{abntex2modelo}.

\section{DIVISÃO DE DOCUMENTOS}
\label{sec:divisao}

A principal divisão de uma monografia é chamada de seção primária (conhecida por capítulo)\cite{NBR6024:2012}. Para criar um capítulo, deve-se utilizar o comando \verb!\chapter{}!. Por exemplo, este capítulo foi definido conforme o comando:

\begin{verbatim}
    \chapter{COMANDOS LATEX}
\end{verbatim}

Um capítulo é dividido em seções, chamadas de secundárias. Uma seção secundária é criada utilizando o comando

\begin{verbatim}
    \section{<nome da seção>} 
\end{verbatim}

Da mesma maneira que um capítulo é referenciado, qualquer seção (não importa o seu nível) pode ser referenciada utilizando o \verb!autoref{}!

% ---
\subsection{Seção Terciária}
\label{sec:secaoTerciaria}
% ---

 Uma seção secundária pode ser dividida em mais seções, chamadas de terciária. Esta é a \autoref{sec:secaoTerciaria}. Uma seção terciária é criada utilizando o comando 

\begin{verbatim}
    \subsection{<nome da subseção>} 
\end{verbatim}

\subsubsection{Seção Quaternária}
\label{sec:secaoQuaternaria}

A subdivisão do texto a partir de uma seção terciária é chamada de seção quaternária. Uma seção quaternária é criada utilizando o comando 

\begin{verbatim}
    \subsubsection{<nome da subseção>} 
\end{verbatim}

Não se recomenda a divisão do texto a partir de uma seção quaternária. 

\section{REFERENCIAR SEÇÕES, CAPÍTULOS E OUTROS ELEMENTOS}

Em um texto técnico, é muito comum que capítulos ou seções sejam referenciados. Além de existir uma formatação própria para realizar o referenciamento, deve-se garantir que a numeração esteja sempre correta (a numeração das divisões do documento podem mudar durante a sua construção). 

De modo a eliminar qualquer dúvida sobre os termos a serem utilizados ao referenciar, somente as palavras apêndice, anexo, capítulo e seção devem ser utilizadas para referenciar qualquer divisão do documento. Não existe lugar para o uso de subseção ou troca do termo capítulo por seção, ou vice-versa. Além disso, como o termo é seguido por um número identificador, a primeira letra deve estar em maiúscula. 

Para que a divisão do documento seja referenciada conforme as normas, deve-se adicionar o comando de definição de rótulo \verb!\label{}!. Por exemplo, este capítulo foi definido conforme os comandos:

\begin{verbatim}
     \chapter{COMANDOS LATEX}
     \label{cap:comandos}
\end{verbatim}

Para referenciar uma divisão, utiliza-se o comando \verb!autoref{}! com o rótulo designado. Além de realizar uma referência ao número da divisão, o comando produz o termo apropriado (depende do tipo de título). 

No caso deste capítulo, o comando \verb!\autoref{cap:comandos}! produzirá o texto \autoref{cap:comandos}. Nota-se que o capítulo deve possuir um rótulo definido como "cap:estrutura". No caso de seções, o texto produzido é "\autoref{sec:secaoTerciaria}." ou "\autoref{sec:secaoQuaternaria}." Em todos os casos, o termo é gerado com a primeira letra maiúscula conforme a formatação da  \ac{ABNT}. 

No caso de múltiplas seções serem referenciadas, deve-se utilizar o termo da divisão seguido pelo comando \verb!\ref{}! para a inserção do número. Por exemplo,

"As Seções \ref{sec:enum} e \ref{sec:divisao} tratam..." 

"Os Capítulos \ref{cap:configuracao} e \ref{cap:consideracoes} apresentam..."

É importante que o termo tenha a primeira letra em maiúsculo. 


\section{ILUSTRAÇÕES}

As ilustrações incluem figuras, tabelas, quadros e algoritmos. Quando utilizadas, estas ilustrações produzem uma lista na parte pré-textual do documento. 

\subsection{Figuras}

As ilustrações devem sempre ter numeração contínua e única em todo o documento:

\begin{citacao}
	Qualquer que seja o tipo de ilustração, sua identificação aparece na parte
	superior, precedida da palavra designativa (desenho, esquema, fluxograma,
	fotografia, gráfico, mapa, organograma, planta, quadro, retrato, figura,
	imagem, entre outros), seguida de seu número de ordem de ocorrência no texto,
	em algarismos arábicos, travessão e do respectivo título. Após a ilustração, na
	parte inferior, indicar a fonte consultada (elemento obrigatório, mesmo que
	seja produção do próprio autor), legenda, notas e outras informações
	necessárias à sua compreensão (se houver). A ilustração deve ser citada no
	texto e inserida o mais próximo possível do trecho a que se
	refere \cite{NBR14724:2011}.
\end{citacao}

A \autoref{fig:exemploFigura} mostra um exemplo de figura, com os principais elementos descritos pela norma. A referência a uma figura deve utilizar o comando \verb!autoref{}!. O comando \texttt{UCSfigura} garante o alinhamento da fonte no lado esquerdo da figura.  

    \begin{figure}[!ht]
    	\caption{Exemplo de figura}
    	\centering 
    	\label{fig:exemploFigura}
    	\UCSfigura {}{ 
    		\includegraphics{figuras/estruturaTCC.png}
    		}{
    		\fonte{Adaptado de \citeonline{guiaUCS}.}
    	}
    \end{figure}   

Algumas das figuras podem ser formadas por várias figuras e as mesmas referenciadas no texto. O pacote \texttt{subfigure}  permite inserir e identificar cada uma destas figuras. A \autoref{fig:estrutura} mostra um exemplo de como colocar duas figuras lado a lado com rótulos próprios para referenciamento no texto. A \autoref{fig:estuturaABNT1} aponta para a figura da esquerda e \autoref{fig:estuturaTCC1} a da direita. 


\begin{figure}[!ht]
	\centering
	\caption{Estruturas de trabalhos pela ABNT e o adotado pela Área da Informática da UCS.}
	\label{fig:estrutura} 
	\begin{subfigure}{0.45\textwidth}
		\centering
		\includegraphics[width = \textwidth]{figuras/estruturaABNT.png}
		\caption{Estrutura de Documentos conforme a ABNT}
		\label{fig:estuturaABNT1}
	\end{subfigure}
	\begin{subfigure}{0.45\textwidth}
		\centering
		\includegraphics[width = \textwidth]{figuras/estruturaTCC.png}
		\caption{Estrutura do TCC }
		\label{fig:estuturaTCC1}
	\end{subfigure}
			
	\fonte{O Autor (2021).}
		
\end{figure}

O \autorefanexo{anex:exemploFiguras} apresenta mais exemplos de figuras que utilizam o \texttt{subfigure}. 

No caso de imagens retiradas da internet, não é necessário adicionar a URL do site na lista de referências do trabalho. Como o objetivo é somente ilustrar, pode-se referenciar o website. A \autoref{fig:exemploImagemInternet} mostra um exemplo de citação da fonte de internet. 

\begin{figure}[!ht]
	\caption{Exemplo de figura com fonte da internet.}
	\centering 
	\label{fig:exemploImagemInternet}
	\UCSfigura {}{ 
		\includegraphics[width = 0.45\textwidth]{figuras/biblio-central.jpg}
		}{ 	\fonte{Disponível em: <https://www.ucs.br/site/biblioteca/biblioteca-central/>. Acesso em 21 mai. 2021} 
		}
\end{figure} 

% ---
\subsection{Tabelas}
% ---
As tabelas apresentam informações numéricas tratadas
estatisticamente \cite{ibge1993}. A estrutura da tabela é constituída de traços (retas perpendiculares) e delimitada, na parte superior e na parte inferior por traços horizontais paralelos. Não delimitar (fechar) por traços verticais os extremos da tabela à direita e à esquerda. A \autoref{tab:ibge} é um exemplo de tabela conforme o padrão IBGE requerido pelas normas da ABNT para documentos técnicos e acadêmicos. Além da fonte na parte inferior da tabela, o exemplo traz como se insere uma nota e anotações, quando necessários. O comando \texttt{IBGEtab} garante o alinhamento da fonte no lado esquerdo da tabela.  

\begin{table}[!ht]
	\caption{\label{tab:ibge} Um Exemplo de tabela alinhada que pode ser longa ou curta}%
	\IBGEtab{}{
		\begin{tabular}{lccc}
			\toprule
			\textbf{Pessoa} & \textbf{Idade} & \textbf{Peso} & \textbf{Altura} \\
			\midrule \midrule
			Marcos          & 26             & 68            & 178             \\
			Ivone           & 22             & 57            & 162             \\ 
			...             & ...            & ...           & ...             \\
			Sueli           & 40             & 65            & 153             \\ 
			\bottomrule
		\end{tabular}
		}{
		\fonte{O Autor (2021).}
		\nota{esta é uma nota, que diz que os dados são baseados na
		regressão linear.}
		\nota[Anotações]{uma anotação adicional, seguida de várias outras.}
	}
\end{table}
    
O \autoref{cap:exemploTabelas} apresenta mais exemplos de tabelas, incluindo uma tabela que ultrapassa uma página (formatação própria). 

% ---
\subsection{Quadros}
% ---

A formatação do quadro é similar à tabela, mas deve ter suas laterais fechadas e conter as linhas horizontais. O \autoref{quad:exemplo} mostra um exemplo do uso de quadro no texto. O comando \texttt{UCSQuadro} garante o alinhamento da fonte no lado esquerdo do quadro.   

\begin{quadro}[!ht]
	\caption{\label{quad:exemplo} Exemplo de quadro}
	\UCSQuadro{}{%
	\begin{tabular}{|c|l|c|}
		\hline
		\textbf{Norma} & \textbf{Título} & \textbf{Ano de Publicação} \\
		\hline
		NBR 14724  & Trabalho Acadêmico    & 2011  \\ \hline
		NBR 10520   & Citações   & 2002   \\ \hline
		NBR 6023  & Referências & 2018 (Versão Corrigida 2:2020)  \\ \hline
		NBR 6027   & Sumário    & 2012  \\ \hline
		NBR 6028  & Resumo e Abstract     & 2003   \\ \hline
		NBR 6024  & Numeração progressiva das seções    & 2012 \\ \hline
		NBR 6034  & Índice    & 2004   \\ \hline
	\end{tabular}
	}{
	\fonte{O Autor (2021).
	}
}\end{quadro}

O \autoref{cap:exemploQuadros} apresenta mais exemplos de quadros, incluindo um quadro que ultrapassa uma página (formatação própria). 

\subsection{Algoritmos/Códigos Fontes}    
    
Não existe uma norma que padronize a formatação de algoritmos ou código fontes em trabalhos técnicos. Neste modelo, utiliza-se o pacote \textit{Listings} \footnote{https://ctan.org/pkg/listings} para a formatação. 

Para facilitar a utilização deste recurso, a linguagem padrão de formatação é C, conforme consta no arquivo TCC.tex e ilustrado na  \ref{fig:linguagem}. Caso não seja C a linguagem mais utilizada, substitua pelo nome da linguagem a ser utilizada conforme o padrão definido na documentação do pacote \textit{Listings}.

\begin{figure}[!ht]
\centering 
\caption{Configuração do pacote \textit{Listings}}
\label{fig:linguagem}
\UCSfigura{}{%
    \includegraphics[scale=0.70]{figuras/linguagem.png}
}{%
 \fonte{O Autor (2021).}%
 }
\end{figure}

Um exemplo é apresentado no Algoritmo \ref{codigo:exemplo1}. Nota que para a configuração deste algoritmo, foram utilizadas somente 2 campos: \textit{caption} e \textit{label}. 

\begin{lstlisting}[ caption={Exemplo de código em C},
                    label={codigo:exemplo1}]
#include <stdio.h>
int main() {    
    int numero1, numero2, soma;
    
    printf("Digite 2 numeros: ");
    scanf("%d %d", &numero1, &numero2);

    // Calcular a soma
    soma = numero1 + numero2;      
    
    printf("%d + %d = %d", numero1, numero2, soma);
    return 0;
}
\end{lstlisting}
\fonte{O Autor (2021)}

No caso de precisar adicionar um código que não seja da linguagem padrão, pode-se fazer utilizando o campo \textit{language} ( \verb!\begin{lstlisting}[language=Python,...]!). Um exemplo de um código na linguagem Python é mostrado no Algoritmo \ref{codigo:exemplo2}).

\begin{lstlisting}[language=Python,
                    caption={Exemplo de código em Python},
                    label={codigo:exemplo2}]
n = int(input("Digite o valor de n: "))
fatorial = 1
while (n > 0):
    fatorial = fatorial * n
    n -= 1
print(fatorial)
\end{lstlisting}
\fonte{O Autor (2021)}

\section{SIGLAS}
\label{sec:siglas}

Para utilizar siglas, primeiro defina a sigla no arquivo "acronimos.tex"~ na pasta "1-pre-textuais"~com o comando 

 \begingroup
\fontsize{10pt}{12pt}\selectfont
   \verb!\acro{PC}{Personal Computer}!
\endgroup
    
O comando deve ser colocado no arquivo de tal maneira que mantenha a ordem alfabética. O \textit{LaTex} garante somente se ela deve ser mostrada ou não na lista.

Para referenciar a sigla, deve-se utilizar o comando:

\begingroup
\fontsize{10pt}{12pt}\selectfont
\verb!\ac{PC}! \newline
\endgroup

A primeira vez que o comando é usado para uma determinada sigla, aparece o significado por extenso da sigla com a sua abreviação em seguida, desta maneira --- \ac{PC}. A partir da segunda vez que o comando para uma determinada sigla é usado, aparace apenas a sigla --- \ac{PC}. Assim, o autor nunca terá que se preocupar com a regra de sempre colocar a sigla por extenso na primeira vez. 

No caso de utilizar o plural da sigla, utilize-o com a letra s após a referência, como em \acp{PC}. Para referenciar a sigla no plural, deve-se utilizar o comando:

\begingroup
\fontsize{10pt}{12pt}\selectfont
\verb!\acp{PC}! \newline
\endgroup

É muito comum ver em trabalhos o uso de 's após a sigla. Este tipo de construção é utilizada para o possessivo da língua inglesa. 

\textbf{Não use siglas nos títulos e evite fortemente usar no Resumo. }

% ---
\section{ENUMERAÇÕES: ALÍNEAS e SUB-ALÍNEAS}
\label{sec:enum} 
% ---

Quando for necessário enumerar os diversos assuntos de uma seção que não possua título, esta deve ser subdividida em alíneas \cite[4.2]{NBR6024:2012}:

\begin{alineas}
  \item os diversos assuntos que não possuam título próprio, dentro de uma mesma
  seção, devem ser subdivididos em alíneas; 
  
  \item o texto que antecede as alíneas termina em dois pontos;
  \item as alíneas devem ser indicadas alfabeticamente, em letra minúscula,
  seguida de parêntese. Utilizam-se letras dobradas, quando esgotadas as
  letras do alfabeto;

  \item as letras indicativas das alíneas devem apresentar recuo em relação à
  margem esquerda;

  \item o texto da alínea deve começar por letra minúscula e terminar em
  ponto-e-vírgula, exceto a última alínea que termina em ponto final;

  \item o texto da alínea deve terminar em dois pontos, se houver subalínea;

  \item a segunda e as seguintes linhas do texto da alínea começa sob a
  primeira letra do texto da própria alínea;
  
  \item subalíneas \cite[4.3]{NBR6024:2012} devem ser conforme as alíneas a
  seguir:

  \begin{alineas}
     \item as subalíneas devem começar por travessão seguido de espaço;

     \item as subalíneas devem apresentar recuo em relação à alínea;

     \item o texto da subalínea deve começar por letra minúscula e terminar em
     ponto-e-vírgula. A última subalínea deve terminar em ponto final, se não
     houver alínea subsequente;

     \item a segunda e as seguintes linhas do texto da subalínea começam sob a
     primeira letra do texto da própria subalínea.
  \end{alineas}
  
  \item no \abnTeX\ estão disponíveis os ambientes \texttt{incisos} e
  \texttt{subalineas}, que em suma são o mesmo que se criar outro nível de
  \texttt{alineas}, como nos exemplos a seguir:
  
  \begin{incisos}
    \item \textit{Um novo inciso em itálico};
  \end{incisos}
  
  \item Alínea em \textbf{negrito}:
  
  \begin{subalineas}
    \item \textit{Uma subalínea em itálico};
    \item \underline{\textit{Uma subalínea em itálico e sublinhado}}; 
  \end{subalineas}
  
  \item Última alínea com \emph{ênfase}.
  
\end{alineas}

É possível modificar a enumeração para utilizar outras sequências. Deve-se colocar o novo símbolo entre colchetes após o comando \verb=\item[]=. Um exemplo de enumeração alterando a sequência pode ser definido da seguinte forma:

\begin{alineas}
	\item[1.] primeiro item
	\item[2.] segundo item
	\begin{alineas}
		\item[a)] primeiro subalínea 
		\item[b)] segundo subalínea
	\end{alineas}
	\item[3.] terceiro item
\end{alineas}

Existem diversos símbolos pré-definidos que podem ser utilizados para a enumeração:

\begin{alineas}
	\item[$\times$] primeiro item
	\item[$\bullet$] segundo item
	\item[$\diamond$] terceiro item
	\item[$\cdot$] quarto item
	\item[$\ast$] quinto item
	\item[$\circ$] sexto item
	\item[$-$] sétimo item
\end{alineas}

\section{CITAÇÕES} 

As citações são menções de informações extraídas de outras fontes \cite{NBR10520:2002}. Elas podem ser uma transcrição textual de parte da obra (citação direta) ou um texto que transcreve a informação com as palavras do autor do trabalho (citação indireta). 
No caso de citações indiretas, existem duas formas de realizar a citação:

\begin{itemize}
	\item \textbf{Início ou meio da frase}: deve-se citar o autor com a primeira letra em maiúsculo e colocar em seguida, entre parênteses o ano. Para esta construção, utiliza-se o comando \verb=\citeonline{}=. Por exemplo,  \citeonline{Acevedo2013}. No caso de querer adicionar o intervalo de páginas, o comando é  \verb=\citeonline[p. 10--15]{key}=. Por exemplo, \citeonline[pp. 21--30]{Acevedo2013}. No caso de somente uma página, o comando é  \verb=\citeonline[p. 10]{key}=. Por exemplo, \citeonline[pp. 21]{Acevedo2013}.
	          
	\item \textbf{Final da frase}: deve-se citar entre parênteses o autor em letras maiúsculas e colocar em seguida, entre parênteses o ano. Para esta construção, utiliza-se o comando \verb=\cite{}=. Por exemplo, \cite{Acevedo2013}. No caso de querer adicionar o intervalo de páginas, o comando é \verb=\cite[p. 10--15]{key}=. Por exemplo, \cite[pp. 21--30]{Acevedo2013}. No caso de somente uma página, o comando é  \verb=\cite[p. 10]{key}=. Por exemplo, \cite[p. 21]{Acevedo2013}.
\end{itemize}

Para realizar a citação direta de um texto longo (mais de 3 linhas), deve-se utilizar o seguinte comando: \verb=\begin{citacao} trecho transcrito \end{citacao}=. Por exemplo, 
\begin{citacao}
	As citações diretas, no texto, com mais de três linhas, devem ser destacadas com recuo de 4 cm da margem esquerda, com letra menor que a do texto utilizado e sem as aspas. No caso de documentos datilografados, deve-se observar apenas
	o recuo \cite[p. 2]{NBR10520:2002}.
\end{citacao}

No caso de não ter acesso à publicação original, é possível realizar a citação de uma citação, ou seja, a transcrição direta ou indireta de um texto ao qual não se teve acesso ao original. Existe uma formatação própria para realizar este tipo de citação utilizando o termo apud ("citado por"). A citação é composta pelo nome do autor original (ao qual não se teve acesso à obra e não deve constar nas referências do trabalho), o termo apud e o nome do autor da obra consultada (a qual deve constar nas referências do trabalho). Por exemplo, no Guia de Trabalhos Acadêmicos da UCS (\cite{guiaUCS}) existe a menção da norma 12225 da ABNT. A sua citação pode ser realizada da seguinte maneira:

\begingroup
\begin{adjustwidth}{2cm}{2cm}
    \noindent
	As informações devem ser impressas conforme a norma ABNT NBR 12225:2004 (\citeauthor{nbr12225:2004}, \citeyear{nbr12225:2004}\footnote{ASSOCIAÇÃO BRASILEIRA DE NORMAS TÉCNICAS.\textbf{NBR 10520}: Informação e documentação — lombada — apresentação. Rio de Janeiro, 2004. 3 p.} apud \citeauthor{guiaUCS}, \citeyear{guiaUCS}). 
	
	\centering ou
	
	\justifying
	\noindent\citeauthor{nbr12225:2004}\footnote{ASSOCIAÇÃO BRASILEIRA DE NORMAS TÉCNICAS.\textbf{NBR 10520}: Informação e documentação — lombada — apresentação. Rio de Janeiro, 2004. 3 p.} ( \citeyear{nbr12225:2004} apud \citeauthor{guiaUCS}, \citeyear{guiaUCS}) define as normas de como as informações devem ser impressas na lombada. 
\end{adjustwidth}
\endgroup

Em ambos os exemplos, o texto da referência original deve ser colocada no rodapé. De modo a reduzir a digitação, os dados da referência original podem ser colocados no arquivo de bibliografias (referencias.bib), mas utilizando o identificador @hidden (para que não faça parte da lista de referências):

\begingroup
\fontsize{10pt}{12pt}\selectfont
\begin{verbatim}
      @hidden{NBR12225:2004, 
          Address = {Rio de Janeiro},  
          Organization = {ASSOCIAÇÃO BRASILEIRA DE NORMAS TÉCNICAS}, 
          Pages = 3,  
          Subtitle = {Informação e documentação - lombada - apresentação},
          Title = {{NBR} 12225}, 
          Year = 2004
       }  
\end{verbatim}
\endgroup

% ---
\section{EXPRESSÕES MATEMÁTICAS}
\label{sec:expressoes}
% ---

As expressões matemáticas devem ser escritas utilizando ambiente \texttt{equation} para serem numeradas (para que sejam referenciadas):

\begin{equation}\label{eq:exemplo1}
\hat{c} = \underset{c}{\arg\max}  P(\textbf{x}|c).
\end{equation}

A referência a uma equação deve utilizar o comando \verb!autoref{}!. A \autoref{eq:exemplo1} é um exemplo de referência. 

No caso de uma expressão não necessitar ser numerada, é possível usar colchetes para indicar o início de uma expressão:

\[
\left|\sum_{i=1}^n a_ib_i\right|
\le
\left(\sum_{i=1}^n a_i^2\right)^{1/2}
\left(\sum_{i=1}^n b_i^2\right)^{1/2}
\]

Se a expressão matemática puder estar na mesma linha do texto, pode-se escrever expressões matemáticas entre cifrões (\$), como em $ \lim_{x \to \infty} \exp(-x) = 0 $.

Consulte mais informações sobre expressões matemáticas em
\url{https://github.com/abntex/abntex2/wiki/Referencias}.

\section{FORMATAÇÃO DO FONTE}

Para uma formatação do \textit{LaTex}, pode-se utilizar o formatador online do C. Albert Thompson  \footnote{https://c.albert-thompson.com/latex-pretty/}. Este site realiza a indentação do código-fonte \textit{LaTex}. 


    \include{2-textuais/4-Referencias}
    \chapter{DICAS GERAIS}
\label{cap:dicas}
	
Este capítulo tem por objetivo fornecer orientações para a construção da monografia. A \autoref{sec:capitulos} descreve algumas orientações sobre a construção dos capítulos. 

\section{CAPÍTULOS} 
\label{sec:capitulos} 

Todo trabalho de conclusão possui os elementos textuais Introdução e Conclusões. A questão que resta é como estruturar o trabalho entre estes dois capítulos. Nesta seção, serão discutidos algumas orientações sobre como um capítulo pode ser construído. 

Em uma monografia, espera-se que o texto esteja organizado de tal forma que o leitor consiga acompanhar e compreender o que está sendo feito e porquê. Assim, geralmente o desenvolvimento está distribuído em capítulos que tratem do referencial teórico, proposta e experimentos/resultados/discussão. Este último aparece na monografia da segunda disciplina de trabalho de conclusão. 

Em algumas situações, o referencial teórico pode aparecer em dois capítulos devido a sua complexidade ou amplitude. Por exemplo, pode ser que um capítulo forneça uma visão geral da área e as respectivas definições enquanto o o outro trate das características do problema específico do trabalho. O objetivo de dividir em mais de um capítulo é permitir um texto mais conciso e não maçante para o leitor. O texto deve sempre considerar que a compreensão do texto pode ser gradual e acumulativa. 

Um capítulo é formado por texto que apresenta algum tipo de relação. Por exemplo, em um capítulo de referencial teórico, espera-se que o seu conteúdo seja formado por definições e relações com outros trabalhos afins. Em um capítulo que apresente uma proposta de solução, o texto deve tratar somente dos elementos desta solução. Para facilitar a leitura, inicie o capítulo explicando o motivo ou objetivo do mesmo. Além disso, pode-se descrever a estrutura dos capítulos apresentando cada uma das seções secundárias (como é feito no início deste capítulo). Assim, evita-se o uso de frases que apresentam a próxima seção. 

Uma seção que é geralmente incluída no final dos capítulos é a de considerações finais. Esta seção tem por objetivo desenhar conclusões sobre o que foi tratado no texto. Ela auxilia na leitura, pois traz os principais pontos do capítulo que serão considerados no(s) próximo(s) capítulo(s). Mas um erro que muitos cometem nesta seção é a de introduzir o que será tratado no próximo capítulo. É necessário lembrar que a seção de estrutura do trabalho na introdução e o texto inicial do capítulo já realizam este papel de explicar o que cada capítulo traz. 

\section{PARÁGRAFOS}

O parágrafo é a unidade de discurso de um texto \cite{martins2019}. É a divisão do texto em partes menores, que determinam um enfoque. Portanto, uma seção deve ser composta por parágrafos e não frases soltas. 

O parágrafo possui partes distintas\cite{martins2019}. O tópico frasal é a ideia principal do parágrafo. Exprime a ideia principal do parágrafo. Em seguida, frases que dão suporte ou explicam em maiores detalhes o tópico frasal são apresentadas. É possível ainda ter uma conclusão que resuma o parágrafo ou exponha o ponto de maior interesse ou um elemento relacionador que estabeleça o encadeamento entre os parágrafos. Toda vez que trocar o enfoque do parágrafo, deve-se criar um parágrafo. 

Como a primeira frase de um parágrafo estabelece a principal ideia de cada parágrafo, deve ser possível compreender o texto da seção a partir de cada frase. As demais frases de cada parágrafo devem somente dar mais detalhes sobre a ideia principal. Por isso, um parágrafo não deve ser iniciado apresentando uma ilustração.  

\section{APÊNDICES X ANEXOS}

Uma confusão muito comum em trabalhos é o uso de apêndices e anexos. O apêndice "é o texto ou documento elaborado pelo próprio autor, de modo a complementar o texto principal e é apresentado no final do trabalho" \cite{guiaUCS}. O anexo é "destinado à inclusão de materiais não 
elaborados pelo próprio autor, como cópias de artigos, manuais, fôlderes, balancetes, etc., visando a dar suporte à argumentação, fundamentação, ilustração ou comprovação" \cite{guiaUCS}. 

O apêndice e o anexo são elementos opcionais. Somente os títulos (seções primárias) destes elementos devem constar no sumário. Caso tais elementos possuírem ilustrações, as mesmas não devem constar nas respectivas listas da monografia. 

\section{ILUSTRAÇÕES}

Uma boa forma de explicar algo é usando ilustrações (desenhos, esquemas, fluxogramas, fotografias, gráficos, mapas, organogramas, plantas, quadros, retratos e outros). Uma ilustração por si só não faz sentido em um texto, ou seja, a ilustração deve estar sempre relacionada a um tópico do texto. 

Toda ilustração deve ser citada no texto e explicada. O texto deve indicar para o leitor o momento certo para olhar ou analisar a ilustração. Por isso, em algumas situações, deve-se explicar a ilustração. 

Se você entender que as ilustrações estão para auxiliar na compreensão do texto, isso vai facilitar o que deve ser escrito no texto ao referenciar a ilustração. 

\section{EQUAÇÕES}

A primeira consideração na utilização de equações é que elas são parte do texto. Isto é, elas podem ser o meio ou o fim de uma frase e, portanto, deve ter a pontuação definida. Por exemplo, pode-se utilizar a fórmula da relatividade $E = mc^2$ como parte da frase (deve ser limitada por cifrões. Ou ainda, pode-se utilizá-las de forma separada (em nova linha) para dar destaque e também numerá-la: 
\begin{citacao}
Este método normaliza cada variável utilizando a equação
\begin{equation}\label{eq:exemplo2}
Z=\frac{X-\mu}{\sigma},
\end{equation}

onde $\mu$ e $\sigma$ são a média e o desvio padrão estimados a partir do conjunto de vetores de características denotado por $X$.

Com base no teorema descrito, a \autoref{eq:exemplo2} pode ser reescrita...

\end{citacao}

Nota-se que após a equação, utiliza-se a vírgula como continuação da frase. É importante que todas as variáveis da equação sejam explicadas antes ou depois dela. Para que o texto após a equação comece na margem esquerda, não deve haver uma linha em branco após a equação.

Deve-se manter uma utilização consistente de variáveis ao longo do texto. NUNCA reutilize variáveis para definições diferentes. 





    \include{2-textuais/6-Conclusoes}
\end{verbatim}
\endgroup

A numeração no nome do arquivo é para garantir uma ordenação dos arquivos. Caso o nome do arquivo seja alterado, deve-se manter a numeração para garantir a ordem dos capítulos na visualização. 


    \chapter{COMANDOS LATEX}
\label{cap:comandos}

Este capítulo tem por objetivo descrever os comandos \textit{LaTex} para a confecção da monografia. A \autoref{sec:divisao} descreve como é a divisão de documentos em conformidade com a ABNT NBR 6024 \cite{NBR6024:2012}. 

Consulte os manuais do \abnTeX\
\cite{abntex2classe,abntex2cite,abntex2cite-alf} para maiores detalhes sobre o uso deste template. Exemplos e informações adicionais de uso de \abnTeX\ e de \LaTeX podem ser encontrados em \citeonline{abntex2modelo}.

\section{DIVISÃO DE DOCUMENTOS}
\label{sec:divisao}

A principal divisão de uma monografia é chamada de seção primária (conhecida por capítulo)\cite{NBR6024:2012}. Para criar um capítulo, deve-se utilizar o comando \verb!\chapter{}!. Por exemplo, este capítulo foi definido conforme o comando:

\begin{verbatim}
    \chapter{COMANDOS LATEX}
\end{verbatim}

Um capítulo é dividido em seções, chamadas de secundárias. Uma seção secundária é criada utilizando o comando

\begin{verbatim}
    \section{<nome da seção>} 
\end{verbatim}

Da mesma maneira que um capítulo é referenciado, qualquer seção (não importa o seu nível) pode ser referenciada utilizando o \verb!autoref{}!

% ---
\subsection{Seção Terciária}
\label{sec:secaoTerciaria}
% ---

 Uma seção secundária pode ser dividida em mais seções, chamadas de terciária. Esta é a \autoref{sec:secaoTerciaria}. Uma seção terciária é criada utilizando o comando 

\begin{verbatim}
    \subsection{<nome da subseção>} 
\end{verbatim}

\subsubsection{Seção Quaternária}
\label{sec:secaoQuaternaria}

A subdivisão do texto a partir de uma seção terciária é chamada de seção quaternária. Uma seção quaternária é criada utilizando o comando 

\begin{verbatim}
    \subsubsection{<nome da subseção>} 
\end{verbatim}

Não se recomenda a divisão do texto a partir de uma seção quaternária. 

\section{REFERENCIAR SEÇÕES, CAPÍTULOS E OUTROS ELEMENTOS}

Em um texto técnico, é muito comum que capítulos ou seções sejam referenciados. Além de existir uma formatação própria para realizar o referenciamento, deve-se garantir que a numeração esteja sempre correta (a numeração das divisões do documento podem mudar durante a sua construção). 

De modo a eliminar qualquer dúvida sobre os termos a serem utilizados ao referenciar, somente as palavras apêndice, anexo, capítulo e seção devem ser utilizadas para referenciar qualquer divisão do documento. Não existe lugar para o uso de subseção ou troca do termo capítulo por seção, ou vice-versa. Além disso, como o termo é seguido por um número identificador, a primeira letra deve estar em maiúscula. 

Para que a divisão do documento seja referenciada conforme as normas, deve-se adicionar o comando de definição de rótulo \verb!\label{}!. Por exemplo, este capítulo foi definido conforme os comandos:

\begin{verbatim}
     \chapter{COMANDOS LATEX}
     \label{cap:comandos}
\end{verbatim}

Para referenciar uma divisão, utiliza-se o comando \verb!autoref{}! com o rótulo designado. Além de realizar uma referência ao número da divisão, o comando produz o termo apropriado (depende do tipo de título). 

No caso deste capítulo, o comando \verb!\autoref{cap:comandos}! produzirá o texto \autoref{cap:comandos}. Nota-se que o capítulo deve possuir um rótulo definido como "cap:estrutura". No caso de seções, o texto produzido é "\autoref{sec:secaoTerciaria}." ou "\autoref{sec:secaoQuaternaria}." Em todos os casos, o termo é gerado com a primeira letra maiúscula conforme a formatação da  \ac{ABNT}. 

No caso de múltiplas seções serem referenciadas, deve-se utilizar o termo da divisão seguido pelo comando \verb!\ref{}! para a inserção do número. Por exemplo,

"As Seções \ref{sec:enum} e \ref{sec:divisao} tratam..." 

"Os Capítulos \ref{cap:configuracao} e \ref{cap:consideracoes} apresentam..."

É importante que o termo tenha a primeira letra em maiúsculo. 


\section{ILUSTRAÇÕES}

As ilustrações incluem figuras, tabelas, quadros e algoritmos. Quando utilizadas, estas ilustrações produzem uma lista na parte pré-textual do documento. 

\subsection{Figuras}

As ilustrações devem sempre ter numeração contínua e única em todo o documento:

\begin{citacao}
	Qualquer que seja o tipo de ilustração, sua identificação aparece na parte
	superior, precedida da palavra designativa (desenho, esquema, fluxograma,
	fotografia, gráfico, mapa, organograma, planta, quadro, retrato, figura,
	imagem, entre outros), seguida de seu número de ordem de ocorrência no texto,
	em algarismos arábicos, travessão e do respectivo título. Após a ilustração, na
	parte inferior, indicar a fonte consultada (elemento obrigatório, mesmo que
	seja produção do próprio autor), legenda, notas e outras informações
	necessárias à sua compreensão (se houver). A ilustração deve ser citada no
	texto e inserida o mais próximo possível do trecho a que se
	refere \cite{NBR14724:2011}.
\end{citacao}

A \autoref{fig:exemploFigura} mostra um exemplo de figura, com os principais elementos descritos pela norma. A referência a uma figura deve utilizar o comando \verb!autoref{}!. O comando \texttt{UCSfigura} garante o alinhamento da fonte no lado esquerdo da figura.  

    \begin{figure}[!ht]
    	\caption{Exemplo de figura}
    	\centering 
    	\label{fig:exemploFigura}
    	\UCSfigura {}{ 
    		\includegraphics{figuras/estruturaTCC.png}
    		}{
    		\fonte{Adaptado de \citeonline{guiaUCS}.}
    	}
    \end{figure}   

Algumas das figuras podem ser formadas por várias figuras e as mesmas referenciadas no texto. O pacote \texttt{subfigure}  permite inserir e identificar cada uma destas figuras. A \autoref{fig:estrutura} mostra um exemplo de como colocar duas figuras lado a lado com rótulos próprios para referenciamento no texto. A \autoref{fig:estuturaABNT1} aponta para a figura da esquerda e \autoref{fig:estuturaTCC1} a da direita. 


\begin{figure}[!ht]
	\centering
	\caption{Estruturas de trabalhos pela ABNT e o adotado pela Área da Informática da UCS.}
	\label{fig:estrutura} 
	\begin{subfigure}{0.45\textwidth}
		\centering
		\includegraphics[width = \textwidth]{figuras/estruturaABNT.png}
		\caption{Estrutura de Documentos conforme a ABNT}
		\label{fig:estuturaABNT1}
	\end{subfigure}
	\begin{subfigure}{0.45\textwidth}
		\centering
		\includegraphics[width = \textwidth]{figuras/estruturaTCC.png}
		\caption{Estrutura do TCC }
		\label{fig:estuturaTCC1}
	\end{subfigure}
			
	\fonte{O Autor (2021).}
		
\end{figure}

O \autorefanexo{anex:exemploFiguras} apresenta mais exemplos de figuras que utilizam o \texttt{subfigure}. 

No caso de imagens retiradas da internet, não é necessário adicionar a URL do site na lista de referências do trabalho. Como o objetivo é somente ilustrar, pode-se referenciar o website. A \autoref{fig:exemploImagemInternet} mostra um exemplo de citação da fonte de internet. 

\begin{figure}[!ht]
	\caption{Exemplo de figura com fonte da internet.}
	\centering 
	\label{fig:exemploImagemInternet}
	\UCSfigura {}{ 
		\includegraphics[width = 0.45\textwidth]{figuras/biblio-central.jpg}
		}{ 	\fonte{Disponível em: <https://www.ucs.br/site/biblioteca/biblioteca-central/>. Acesso em 21 mai. 2021} 
		}
\end{figure} 

% ---
\subsection{Tabelas}
% ---
As tabelas apresentam informações numéricas tratadas
estatisticamente \cite{ibge1993}. A estrutura da tabela é constituída de traços (retas perpendiculares) e delimitada, na parte superior e na parte inferior por traços horizontais paralelos. Não delimitar (fechar) por traços verticais os extremos da tabela à direita e à esquerda. A \autoref{tab:ibge} é um exemplo de tabela conforme o padrão IBGE requerido pelas normas da ABNT para documentos técnicos e acadêmicos. Além da fonte na parte inferior da tabela, o exemplo traz como se insere uma nota e anotações, quando necessários. O comando \texttt{IBGEtab} garante o alinhamento da fonte no lado esquerdo da tabela.  

\begin{table}[!ht]
	\caption{\label{tab:ibge} Um Exemplo de tabela alinhada que pode ser longa ou curta}%
	\IBGEtab{}{
		\begin{tabular}{lccc}
			\toprule
			\textbf{Pessoa} & \textbf{Idade} & \textbf{Peso} & \textbf{Altura} \\
			\midrule \midrule
			Marcos          & 26             & 68            & 178             \\
			Ivone           & 22             & 57            & 162             \\ 
			...             & ...            & ...           & ...             \\
			Sueli           & 40             & 65            & 153             \\ 
			\bottomrule
		\end{tabular}
		}{
		\fonte{O Autor (2021).}
		\nota{esta é uma nota, que diz que os dados são baseados na
		regressão linear.}
		\nota[Anotações]{uma anotação adicional, seguida de várias outras.}
	}
\end{table}
    
O \autoref{cap:exemploTabelas} apresenta mais exemplos de tabelas, incluindo uma tabela que ultrapassa uma página (formatação própria). 

% ---
\subsection{Quadros}
% ---

A formatação do quadro é similar à tabela, mas deve ter suas laterais fechadas e conter as linhas horizontais. O \autoref{quad:exemplo} mostra um exemplo do uso de quadro no texto. O comando \texttt{UCSQuadro} garante o alinhamento da fonte no lado esquerdo do quadro.   

\begin{quadro}[!ht]
	\caption{\label{quad:exemplo} Exemplo de quadro}
	\UCSQuadro{}{%
	\begin{tabular}{|c|l|c|}
		\hline
		\textbf{Norma} & \textbf{Título} & \textbf{Ano de Publicação} \\
		\hline
		NBR 14724  & Trabalho Acadêmico    & 2011  \\ \hline
		NBR 10520   & Citações   & 2002   \\ \hline
		NBR 6023  & Referências & 2018 (Versão Corrigida 2:2020)  \\ \hline
		NBR 6027   & Sumário    & 2012  \\ \hline
		NBR 6028  & Resumo e Abstract     & 2003   \\ \hline
		NBR 6024  & Numeração progressiva das seções    & 2012 \\ \hline
		NBR 6034  & Índice    & 2004   \\ \hline
	\end{tabular}
	}{
	\fonte{O Autor (2021).
	}
}\end{quadro}

O \autoref{cap:exemploQuadros} apresenta mais exemplos de quadros, incluindo um quadro que ultrapassa uma página (formatação própria). 

\subsection{Algoritmos/Códigos Fontes}    
    
Não existe uma norma que padronize a formatação de algoritmos ou código fontes em trabalhos técnicos. Neste modelo, utiliza-se o pacote \textit{Listings} \footnote{https://ctan.org/pkg/listings} para a formatação. 

Para facilitar a utilização deste recurso, a linguagem padrão de formatação é C, conforme consta no arquivo TCC.tex e ilustrado na  \ref{fig:linguagem}. Caso não seja C a linguagem mais utilizada, substitua pelo nome da linguagem a ser utilizada conforme o padrão definido na documentação do pacote \textit{Listings}.

\begin{figure}[!ht]
\centering 
\caption{Configuração do pacote \textit{Listings}}
\label{fig:linguagem}
\UCSfigura{}{%
    \includegraphics[scale=0.70]{figuras/linguagem.png}
}{%
 \fonte{O Autor (2021).}%
 }
\end{figure}

Um exemplo é apresentado no Algoritmo \ref{codigo:exemplo1}. Nota que para a configuração deste algoritmo, foram utilizadas somente 2 campos: \textit{caption} e \textit{label}. 

\begin{lstlisting}[ caption={Exemplo de código em C},
                    label={codigo:exemplo1}]
#include <stdio.h>
int main() {    
    int numero1, numero2, soma;
    
    printf("Digite 2 numeros: ");
    scanf("%d %d", &numero1, &numero2);

    // Calcular a soma
    soma = numero1 + numero2;      
    
    printf("%d + %d = %d", numero1, numero2, soma);
    return 0;
}
\end{lstlisting}
\fonte{O Autor (2021)}

No caso de precisar adicionar um código que não seja da linguagem padrão, pode-se fazer utilizando o campo \textit{language} ( \verb!\begin{lstlisting}[language=Python,...]!). Um exemplo de um código na linguagem Python é mostrado no Algoritmo \ref{codigo:exemplo2}).

\begin{lstlisting}[language=Python,
                    caption={Exemplo de código em Python},
                    label={codigo:exemplo2}]
n = int(input("Digite o valor de n: "))
fatorial = 1
while (n > 0):
    fatorial = fatorial * n
    n -= 1
print(fatorial)
\end{lstlisting}
\fonte{O Autor (2021)}

\section{SIGLAS}
\label{sec:siglas}

Para utilizar siglas, primeiro defina a sigla no arquivo "acronimos.tex"~ na pasta "1-pre-textuais"~com o comando 

 \begingroup
\fontsize{10pt}{12pt}\selectfont
   \verb!\acro{PC}{Personal Computer}!
\endgroup
    
O comando deve ser colocado no arquivo de tal maneira que mantenha a ordem alfabética. O \textit{LaTex} garante somente se ela deve ser mostrada ou não na lista.

Para referenciar a sigla, deve-se utilizar o comando:

\begingroup
\fontsize{10pt}{12pt}\selectfont
\verb!\ac{PC}! \newline
\endgroup

A primeira vez que o comando é usado para uma determinada sigla, aparece o significado por extenso da sigla com a sua abreviação em seguida, desta maneira --- \ac{PC}. A partir da segunda vez que o comando para uma determinada sigla é usado, aparace apenas a sigla --- \ac{PC}. Assim, o autor nunca terá que se preocupar com a regra de sempre colocar a sigla por extenso na primeira vez. 

No caso de utilizar o plural da sigla, utilize-o com a letra s após a referência, como em \acp{PC}. Para referenciar a sigla no plural, deve-se utilizar o comando:

\begingroup
\fontsize{10pt}{12pt}\selectfont
\verb!\acp{PC}! \newline
\endgroup

É muito comum ver em trabalhos o uso de 's após a sigla. Este tipo de construção é utilizada para o possessivo da língua inglesa. 

\textbf{Não use siglas nos títulos e evite fortemente usar no Resumo. }

% ---
\section{ENUMERAÇÕES: ALÍNEAS e SUB-ALÍNEAS}
\label{sec:enum} 
% ---

Quando for necessário enumerar os diversos assuntos de uma seção que não possua título, esta deve ser subdividida em alíneas \cite[4.2]{NBR6024:2012}:

\begin{alineas}
  \item os diversos assuntos que não possuam título próprio, dentro de uma mesma
  seção, devem ser subdivididos em alíneas; 
  
  \item o texto que antecede as alíneas termina em dois pontos;
  \item as alíneas devem ser indicadas alfabeticamente, em letra minúscula,
  seguida de parêntese. Utilizam-se letras dobradas, quando esgotadas as
  letras do alfabeto;

  \item as letras indicativas das alíneas devem apresentar recuo em relação à
  margem esquerda;

  \item o texto da alínea deve começar por letra minúscula e terminar em
  ponto-e-vírgula, exceto a última alínea que termina em ponto final;

  \item o texto da alínea deve terminar em dois pontos, se houver subalínea;

  \item a segunda e as seguintes linhas do texto da alínea começa sob a
  primeira letra do texto da própria alínea;
  
  \item subalíneas \cite[4.3]{NBR6024:2012} devem ser conforme as alíneas a
  seguir:

  \begin{alineas}
     \item as subalíneas devem começar por travessão seguido de espaço;

     \item as subalíneas devem apresentar recuo em relação à alínea;

     \item o texto da subalínea deve começar por letra minúscula e terminar em
     ponto-e-vírgula. A última subalínea deve terminar em ponto final, se não
     houver alínea subsequente;

     \item a segunda e as seguintes linhas do texto da subalínea começam sob a
     primeira letra do texto da própria subalínea.
  \end{alineas}
  
  \item no \abnTeX\ estão disponíveis os ambientes \texttt{incisos} e
  \texttt{subalineas}, que em suma são o mesmo que se criar outro nível de
  \texttt{alineas}, como nos exemplos a seguir:
  
  \begin{incisos}
    \item \textit{Um novo inciso em itálico};
  \end{incisos}
  
  \item Alínea em \textbf{negrito}:
  
  \begin{subalineas}
    \item \textit{Uma subalínea em itálico};
    \item \underline{\textit{Uma subalínea em itálico e sublinhado}}; 
  \end{subalineas}
  
  \item Última alínea com \emph{ênfase}.
  
\end{alineas}

É possível modificar a enumeração para utilizar outras sequências. Deve-se colocar o novo símbolo entre colchetes após o comando \verb=\item[]=. Um exemplo de enumeração alterando a sequência pode ser definido da seguinte forma:

\begin{alineas}
	\item[1.] primeiro item
	\item[2.] segundo item
	\begin{alineas}
		\item[a)] primeiro subalínea 
		\item[b)] segundo subalínea
	\end{alineas}
	\item[3.] terceiro item
\end{alineas}

Existem diversos símbolos pré-definidos que podem ser utilizados para a enumeração:

\begin{alineas}
	\item[$\times$] primeiro item
	\item[$\bullet$] segundo item
	\item[$\diamond$] terceiro item
	\item[$\cdot$] quarto item
	\item[$\ast$] quinto item
	\item[$\circ$] sexto item
	\item[$-$] sétimo item
\end{alineas}

\section{CITAÇÕES} 

As citações são menções de informações extraídas de outras fontes \cite{NBR10520:2002}. Elas podem ser uma transcrição textual de parte da obra (citação direta) ou um texto que transcreve a informação com as palavras do autor do trabalho (citação indireta). 
No caso de citações indiretas, existem duas formas de realizar a citação:

\begin{itemize}
	\item \textbf{Início ou meio da frase}: deve-se citar o autor com a primeira letra em maiúsculo e colocar em seguida, entre parênteses o ano. Para esta construção, utiliza-se o comando \verb=\citeonline{}=. Por exemplo,  \citeonline{Acevedo2013}. No caso de querer adicionar o intervalo de páginas, o comando é  \verb=\citeonline[p. 10--15]{key}=. Por exemplo, \citeonline[pp. 21--30]{Acevedo2013}. No caso de somente uma página, o comando é  \verb=\citeonline[p. 10]{key}=. Por exemplo, \citeonline[pp. 21]{Acevedo2013}.
	          
	\item \textbf{Final da frase}: deve-se citar entre parênteses o autor em letras maiúsculas e colocar em seguida, entre parênteses o ano. Para esta construção, utiliza-se o comando \verb=\cite{}=. Por exemplo, \cite{Acevedo2013}. No caso de querer adicionar o intervalo de páginas, o comando é \verb=\cite[p. 10--15]{key}=. Por exemplo, \cite[pp. 21--30]{Acevedo2013}. No caso de somente uma página, o comando é  \verb=\cite[p. 10]{key}=. Por exemplo, \cite[p. 21]{Acevedo2013}.
\end{itemize}

Para realizar a citação direta de um texto longo (mais de 3 linhas), deve-se utilizar o seguinte comando: \verb=\begin{citacao} trecho transcrito \end{citacao}=. Por exemplo, 
\begin{citacao}
	As citações diretas, no texto, com mais de três linhas, devem ser destacadas com recuo de 4 cm da margem esquerda, com letra menor que a do texto utilizado e sem as aspas. No caso de documentos datilografados, deve-se observar apenas
	o recuo \cite[p. 2]{NBR10520:2002}.
\end{citacao}

No caso de não ter acesso à publicação original, é possível realizar a citação de uma citação, ou seja, a transcrição direta ou indireta de um texto ao qual não se teve acesso ao original. Existe uma formatação própria para realizar este tipo de citação utilizando o termo apud ("citado por"). A citação é composta pelo nome do autor original (ao qual não se teve acesso à obra e não deve constar nas referências do trabalho), o termo apud e o nome do autor da obra consultada (a qual deve constar nas referências do trabalho). Por exemplo, no Guia de Trabalhos Acadêmicos da UCS (\cite{guiaUCS}) existe a menção da norma 12225 da ABNT. A sua citação pode ser realizada da seguinte maneira:

\begingroup
\begin{adjustwidth}{2cm}{2cm}
    \noindent
	As informações devem ser impressas conforme a norma ABNT NBR 12225:2004 (\citeauthor{nbr12225:2004}, \citeyear{nbr12225:2004}\footnote{ASSOCIAÇÃO BRASILEIRA DE NORMAS TÉCNICAS.\textbf{NBR 10520}: Informação e documentação — lombada — apresentação. Rio de Janeiro, 2004. 3 p.} apud \citeauthor{guiaUCS}, \citeyear{guiaUCS}). 
	
	\centering ou
	
	\justifying
	\noindent\citeauthor{nbr12225:2004}\footnote{ASSOCIAÇÃO BRASILEIRA DE NORMAS TÉCNICAS.\textbf{NBR 10520}: Informação e documentação — lombada — apresentação. Rio de Janeiro, 2004. 3 p.} ( \citeyear{nbr12225:2004} apud \citeauthor{guiaUCS}, \citeyear{guiaUCS}) define as normas de como as informações devem ser impressas na lombada. 
\end{adjustwidth}
\endgroup

Em ambos os exemplos, o texto da referência original deve ser colocada no rodapé. De modo a reduzir a digitação, os dados da referência original podem ser colocados no arquivo de bibliografias (referencias.bib), mas utilizando o identificador @hidden (para que não faça parte da lista de referências):

\begingroup
\fontsize{10pt}{12pt}\selectfont
\begin{verbatim}
      @hidden{NBR12225:2004, 
          Address = {Rio de Janeiro},  
          Organization = {ASSOCIAÇÃO BRASILEIRA DE NORMAS TÉCNICAS}, 
          Pages = 3,  
          Subtitle = {Informação e documentação - lombada - apresentação},
          Title = {{NBR} 12225}, 
          Year = 2004
       }  
\end{verbatim}
\endgroup

% ---
\section{EXPRESSÕES MATEMÁTICAS}
\label{sec:expressoes}
% ---

As expressões matemáticas devem ser escritas utilizando ambiente \texttt{equation} para serem numeradas (para que sejam referenciadas):

\begin{equation}\label{eq:exemplo1}
\hat{c} = \underset{c}{\arg\max}  P(\textbf{x}|c).
\end{equation}

A referência a uma equação deve utilizar o comando \verb!autoref{}!. A \autoref{eq:exemplo1} é um exemplo de referência. 

No caso de uma expressão não necessitar ser numerada, é possível usar colchetes para indicar o início de uma expressão:

\[
\left|\sum_{i=1}^n a_ib_i\right|
\le
\left(\sum_{i=1}^n a_i^2\right)^{1/2}
\left(\sum_{i=1}^n b_i^2\right)^{1/2}
\]

Se a expressão matemática puder estar na mesma linha do texto, pode-se escrever expressões matemáticas entre cifrões (\$), como em $ \lim_{x \to \infty} \exp(-x) = 0 $.

Consulte mais informações sobre expressões matemáticas em
\url{https://github.com/abntex/abntex2/wiki/Referencias}.

\section{FORMATAÇÃO DO FONTE}

Para uma formatação do \textit{LaTex}, pode-se utilizar o formatador online do C. Albert Thompson  \footnote{https://c.albert-thompson.com/latex-pretty/}. Este site realiza a indentação do código-fonte \textit{LaTex}. 


    \include{2-textuais/4-Referencias}
    \chapter{DICAS GERAIS}
\label{cap:dicas}
	
Este capítulo tem por objetivo fornecer orientações para a construção da monografia. A \autoref{sec:capitulos} descreve algumas orientações sobre a construção dos capítulos. 

\section{CAPÍTULOS} 
\label{sec:capitulos} 

Todo trabalho de conclusão possui os elementos textuais Introdução e Conclusões. A questão que resta é como estruturar o trabalho entre estes dois capítulos. Nesta seção, serão discutidos algumas orientações sobre como um capítulo pode ser construído. 

Em uma monografia, espera-se que o texto esteja organizado de tal forma que o leitor consiga acompanhar e compreender o que está sendo feito e porquê. Assim, geralmente o desenvolvimento está distribuído em capítulos que tratem do referencial teórico, proposta e experimentos/resultados/discussão. Este último aparece na monografia da segunda disciplina de trabalho de conclusão. 

Em algumas situações, o referencial teórico pode aparecer em dois capítulos devido a sua complexidade ou amplitude. Por exemplo, pode ser que um capítulo forneça uma visão geral da área e as respectivas definições enquanto o o outro trate das características do problema específico do trabalho. O objetivo de dividir em mais de um capítulo é permitir um texto mais conciso e não maçante para o leitor. O texto deve sempre considerar que a compreensão do texto pode ser gradual e acumulativa. 

Um capítulo é formado por texto que apresenta algum tipo de relação. Por exemplo, em um capítulo de referencial teórico, espera-se que o seu conteúdo seja formado por definições e relações com outros trabalhos afins. Em um capítulo que apresente uma proposta de solução, o texto deve tratar somente dos elementos desta solução. Para facilitar a leitura, inicie o capítulo explicando o motivo ou objetivo do mesmo. Além disso, pode-se descrever a estrutura dos capítulos apresentando cada uma das seções secundárias (como é feito no início deste capítulo). Assim, evita-se o uso de frases que apresentam a próxima seção. 

Uma seção que é geralmente incluída no final dos capítulos é a de considerações finais. Esta seção tem por objetivo desenhar conclusões sobre o que foi tratado no texto. Ela auxilia na leitura, pois traz os principais pontos do capítulo que serão considerados no(s) próximo(s) capítulo(s). Mas um erro que muitos cometem nesta seção é a de introduzir o que será tratado no próximo capítulo. É necessário lembrar que a seção de estrutura do trabalho na introdução e o texto inicial do capítulo já realizam este papel de explicar o que cada capítulo traz. 

\section{PARÁGRAFOS}

O parágrafo é a unidade de discurso de um texto \cite{martins2019}. É a divisão do texto em partes menores, que determinam um enfoque. Portanto, uma seção deve ser composta por parágrafos e não frases soltas. 

O parágrafo possui partes distintas\cite{martins2019}. O tópico frasal é a ideia principal do parágrafo. Exprime a ideia principal do parágrafo. Em seguida, frases que dão suporte ou explicam em maiores detalhes o tópico frasal são apresentadas. É possível ainda ter uma conclusão que resuma o parágrafo ou exponha o ponto de maior interesse ou um elemento relacionador que estabeleça o encadeamento entre os parágrafos. Toda vez que trocar o enfoque do parágrafo, deve-se criar um parágrafo. 

Como a primeira frase de um parágrafo estabelece a principal ideia de cada parágrafo, deve ser possível compreender o texto da seção a partir de cada frase. As demais frases de cada parágrafo devem somente dar mais detalhes sobre a ideia principal. Por isso, um parágrafo não deve ser iniciado apresentando uma ilustração.  

\section{APÊNDICES X ANEXOS}

Uma confusão muito comum em trabalhos é o uso de apêndices e anexos. O apêndice "é o texto ou documento elaborado pelo próprio autor, de modo a complementar o texto principal e é apresentado no final do trabalho" \cite{guiaUCS}. O anexo é "destinado à inclusão de materiais não 
elaborados pelo próprio autor, como cópias de artigos, manuais, fôlderes, balancetes, etc., visando a dar suporte à argumentação, fundamentação, ilustração ou comprovação" \cite{guiaUCS}. 

O apêndice e o anexo são elementos opcionais. Somente os títulos (seções primárias) destes elementos devem constar no sumário. Caso tais elementos possuírem ilustrações, as mesmas não devem constar nas respectivas listas da monografia. 

\section{ILUSTRAÇÕES}

Uma boa forma de explicar algo é usando ilustrações (desenhos, esquemas, fluxogramas, fotografias, gráficos, mapas, organogramas, plantas, quadros, retratos e outros). Uma ilustração por si só não faz sentido em um texto, ou seja, a ilustração deve estar sempre relacionada a um tópico do texto. 

Toda ilustração deve ser citada no texto e explicada. O texto deve indicar para o leitor o momento certo para olhar ou analisar a ilustração. Por isso, em algumas situações, deve-se explicar a ilustração. 

Se você entender que as ilustrações estão para auxiliar na compreensão do texto, isso vai facilitar o que deve ser escrito no texto ao referenciar a ilustração. 

\section{EQUAÇÕES}

A primeira consideração na utilização de equações é que elas são parte do texto. Isto é, elas podem ser o meio ou o fim de uma frase e, portanto, deve ter a pontuação definida. Por exemplo, pode-se utilizar a fórmula da relatividade $E = mc^2$ como parte da frase (deve ser limitada por cifrões. Ou ainda, pode-se utilizá-las de forma separada (em nova linha) para dar destaque e também numerá-la: 
\begin{citacao}
Este método normaliza cada variável utilizando a equação
\begin{equation}\label{eq:exemplo2}
Z=\frac{X-\mu}{\sigma},
\end{equation}

onde $\mu$ e $\sigma$ são a média e o desvio padrão estimados a partir do conjunto de vetores de características denotado por $X$.

Com base no teorema descrito, a \autoref{eq:exemplo2} pode ser reescrita...

\end{citacao}

Nota-se que após a equação, utiliza-se a vírgula como continuação da frase. É importante que todas as variáveis da equação sejam explicadas antes ou depois dela. Para que o texto após a equação comece na margem esquerda, não deve haver uma linha em branco após a equação.

Deve-se manter uma utilização consistente de variáveis ao longo do texto. NUNCA reutilize variáveis para definições diferentes. 





    \include{2-textuais/6-Conclusoes}
\end{verbatim}
\endgroup

A numeração no nome do arquivo é para garantir uma ordenação dos arquivos. Caso o nome do arquivo seja alterado, deve-se manter a numeração para garantir a ordem dos capítulos na visualização. 

